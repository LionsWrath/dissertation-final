\documentclass[msc,english,table,xcdraw]{ppgccufmg} 

\usepackage[english]{babel}
\usepackage[utf8]{inputenc}
\usepackage[T1]{fontenc}
\usepackage{type1ec}
\usepackage{graphicx}
\usepackage[
	colorlinks=true,
	linkcolor=blue, %% Cor dos links do sumário
	citecolor=red, %% Cor dos links das citações      
	urlcolor=magenta, %% Cor das urls
]{hyperref}
\usepackage{natbib}

% Added
\usepackage{xcolor}
\usepackage{float}
\usepackage{tikz}
    \usetikzlibrary{decorations.markings}
    \usetikzlibrary{positioning}
    \usetikzlibrary{calc}
\usepackage{subcaption}

\usepackage{multirow}
\usepackage{xstring}
\usepackage{ifthen}
\usepackage{subcaption}

%% Algorithms
\usepackage[ruled,vlined]{algorithm2e}
\SetKwRepeat{Do}{do}{while}

%% Custom packages
\usepackage{colors}
\usepackage{macros}

%\newfloat[chapter]{algoritmo}{lol}{Algoritmo}
\newcommand{\listaalgoritmosname}{Lista de Algoritmos} %% Título da lista
\newlistof{listadealgoritmos}{lol}{\listaalgoritmosname}
\newlistentry{algoritmo}{lol}{0} %% Nome do ambiente de cada algoritmo, e.g., \begin{algoritmo} ... \end{algoritmo}

%% **** Caso não haja nenhuma lista adicional, os comandos acima podem ser apagados. ****
%% ===============================================

\begin{document}

\ppgccufmg{
    autor={Caio Henrique Segawa Tonetti}, 
    titulo={The Token Swap Problem: Polynomial Time Algorithms for Graph Classes and Integer Linear Programming Formulations},
    cidade={Belo Horizonte},
    ano={2024},
    versao={Final}, %% Versão do documento
    orientador={Vinicius Fernandes dos Santos},
    coorientador={Sebastián Urrutia}, 
    %fichacatalografica={fichacatalografica.pdf},
    %folhadeaprovacao={folhadeaprovacao.pdf},
    resumo={resumo.tex},
    abstracten={abstract.tex},
    palavraschave={Teoria dos Grafos, Reconfiguração, Programação Inteira}, 
    keywords={Graph Theory, Reconfiguration, Integer Programming}, 
    dedicatoria={dedicatoria.tex}, %% Arquivo .tex contendo a dedicatória
    agradecimentos={agradecimentos.tex},
    epigrafe={For nothing is self-sufficient, neither in us ourselves nor in things; and if our soul has trembled with happiness and sounded like a harp string just once, all eternity was needed to produce this one event—and in this single moment of affirmation all eternity was called good, redeemed, justified, and affirmed.},
    epigrafeautor={Friedrich Nietzsche},
    listadefiguras={sim}, %% Remova (ou comente) este parâmetro para remover a lista de figuras
    listadetabelas={sim}, %% Remova (ou comente) este parâmetro para remover a lista de tabelas
    listascustomizadas={\listadealgoritmos} %% Lista customizada (e.g., lista de algoritmos). 
}

% Os três comandos seguintes são apenas para gerar texto para ocupar espaço nas
% páginas.
% \newcommand{\dummytxta}{%
% Lorem ipsum dolor sit amet, consectetur adipisicing elit, sed do
% eiusmod tempor incididunt ut labore et dolore magna aliqua. Ut enim ad
% minim veniam, quis nostrud exercitation ullamco laboris nisi ut
% aliquip ex ea commodo consequat. Duis aute irure dolor in
% reprehenderit in voluptate velit esse cillum dolore eu fugiat nulla
% pariatur. Excepteur sint occaecat cupidatat non proident, sunt in
% culpa qui officia deserunt mollit anim id est laborum.\par
% }

% \newcommand{\dummytxtb}{%
% Sed ut perspiciatis unde omnis iste natus error sit voluptatem accusantium
% doloremque laudantium, totam rem aperiam, eaque ipsa quae ab illo inventore
% veritatis et quasi architecto beatae vitae dicta sunt explicabo. Nemo enim
% ipsam voluptatem quia voluptas sit aspernatur aut odit aut fugit, sed quia
% consequuntur magni dolores eos qui ratione voluptatem sequi nesciunt. Neque
% porro quisquam est, qui dolorem ipsum quia dolor sit amet, consectetur,
% adipisci velit, sed quia non numquam eius modi tempora incidunt ut labore et
% dolore magnam aliquam quaerat voluptatem. Ut enim ad minima veniam, quis
% nostrum exercitationem ullam corporis suscipit laboriosam, nisi ut aliquid ex
% ea commodi consequatur? Quis autem vel eum iure reprehenderit qui in ea
% voluptate velit esse quam nihil molestiae consequatur, vel illum qui dolorem
% eum fugiat quo voluptas nulla pariatur?\par
% }

% \newcommand{\dummytxtc}{%
% At vero eos et accusamus et iusto odio dignissimos ducimus qui blanditiis
% praesentium voluptatum deleniti atque corrupti quos dolores et quas molestias
% excepturi sint occaecati cupiditate non provident, similique sunt in culpa qui
% officia deserunt mollitia animi, id est laborum et dolorum fuga. Et harum
% quidem rerum facilis est et expedita distinctio. Nam libero tempore, cum soluta
% nobis est eligendi optio cumque nihil impedit quo minus id quod maxime placeat
% facere possimus, omnis voluptas assumenda est, omnis dolor
% repellendus. Temporibus autem quibusdam et aut officiis debitis aut rerum
% necessitatibus saepe eveniet ut et voluptates repudiandae sint et molestiae non
% recusandae. Itaque earum rerum hic tenetur a sapiente delectus, ut aut
% reiciendis voluptatibus maiores alias consequatur aut perferendis doloribus
% asperiores repellat.\par
% }

% \newcommand{\dummytxt}{\dummytxta\dummytxtb\dummytxtc}

\chapter{Introduction}
\label{chp:intro}

The reconfiguration framework~\citep{Ito:2011} brings the notion of 
\textit{transformation} in computational problems and new questions arises 
from the necessity of understanding these changes under various operation 
and constraints.
A simple version of this problem is sorting a list of numbers by adjacent swaps 
between two elements, in which the swap number is exactly the number of 
pairs that are out of order, being recognized as the number of \textit{
inversions} in a list and two elements are adjacent if they are next to each 
other in the list.
This swap sequence can be calculated in polynomial time by a modified bubble 
sort algorithm~\citep{Knuth:1998}.
Under other conditions, when sorting the lists by \textit{prefix-reversals}, 
an elementary operation that flips a prefix of the list, finding the 
minimum number of flips is \textbf{NP-complete}~\citep{Bulteau:2015}.
This problem is called the pancake sorting problem.
Another reconfiguration problem is the $n$-puzzle, a sliding puzzle on a 
grid of $n$ numbered square tiles with exactly one tile missing.
In this problem, each step can \textit{slide} a tile to any adjacent empty tile 
space to achieve a final sorted configuration state.
Research of the $15$-puzzle version of the $n$-puzzle dates back to the late 
19\textsuperscript{th} century~\citep{Johnson:1879} and is commonly used as an 
introductory problem for modelling heuristics.
Figure~\ref{img:npuzzle} is an example of a $15$-puzzle instance.

Considering the reconfiguration problems, the interest lies in one of the three
following parameters: connectivity, diameter or distance.
This parameters translate, respectively, to the following questions: (a) can any
configuration be reconfigured into another?; (b) what is the maximum number of 
required operation steps for any reconfiguration?; and (c) what is the minimum 
number of operations for any particular reconfiguration?
In the examples aforementioned, the distance between two configurations (the 
initial and the sorted) was being used.
For the interested, more information about most of the known reconfiguration 
problems can be found in the surveys by \citep
{Heuvel:2013,Mouawad:2015,Naomi:2018}.

\addExample{npuzzle.tex}

This dissertation focus on a reconfiguration problem called the {\texttt Token
Swap} problem (TS).
Let $\graphG = \graphPair$ be a graph with $\vertexNumber = \sizeof{\vertexSet}$ 
vertices and $\sizeof{\edgeSet}$ edges, with distinct tokens placed on it's 
vertices. 
The objective is to reconfigure this initial token placement called  
$\defineFunc{\initMap}{\vertexSet}{\vertexSet}$ into the identity token 
placement $\idMap$ that maps every node to itself with minimum distance. 
The reconfiguration must consist of a token swap sequence $S$, being each 
identified by a pair of adjacent graph vertices, meaning that the elementary
operation will be restricted to a swap between two tokens placed on vertices 
that share an edge in the graph.
For the decision version of this problem, the aim is to know if it is possible 
to have a swap sequence $\swapSeq$ that transforms $\initMap$ to $\idMap$ in $k$ 
or less swaps, such that $k \in \mathbb{N}$.
In Figure~\ref{img:ts-example}, a complete example of a TS instance is shown.
The formal mathematical definitions necessary to fully understand this dissertation 
will be given in Section~\ref{sec:intro:prelim}.

\addExample{ts.tex}

Every reconfiguration problem can also be formulated as a reconfiguration 
graph where each node is a possible configuration of the combinatorial
or geometric object and an edge exists between two vertices if and only if 
each of the vertices can be reached in exactly one reconfiguration step from
another~\citep{Naomi:2018}.
In the case of the TS, the reconfiguration graph will be connected as long as
the original graph is also connected.
This connection guarantees that any configuration can be reconfigured to any 
other, as it is possible to take any spanning tree of the original graph and 
move the desired tokens to each corresponding leaf in a way that any chosen 
final configuration is achieved.
This process give us an upper-bound of $\worstCase{\vertexNumber^{2}}$ 
swaps for any instance of the TS problem~\citep{Yamanaka:2015}.
The diameter and distance of this reconfiguration graph is exactly the 
upper-bound of the reconfiguration problem, and the distance between two 
configurations \textemdash i.e., two vertices in the 
reconfiguration graph \textemdash, respectively.
Although the reconfiguration graph of a given graph $\graphG$ could be 
theoretically built from $\graphG$, in practice this is not viable, since, in 
general, the number of configurations is exponential in $\sizeof{\vertexSet}$.

If the TS problem is brought to the realm of permutation group theory, it is
possible to model this problem by a group $\groupPair$, in which every element 
of $F$ is a bijective function representing a possible configuration. 
The binary operation is function composition and every element of $\groupSet$ 
can be represented by the product of finitely many elements of a subset 
$\groupGen$ of $\groupSet$ and their inverses.
The elements of $\groupGen$ are called the generators of the group $\groupPair$ 
and they represent each possible transpositions corresponding to the edges 
of the original graph.
Thereupon, given a TS instance, the Cayley Graph $\cayleyGraph$ of the symmetric 
group will correspond (under isomorphism) exactly to the reconfiguration graph 
of the original problem, the distance between two configurations will be the 
shortest path between those two vertices and the worst case will match the 
diameter of this graph.
The shortest path in a Cayley Graph, also known as the {\texttt Minimum Length
Generator Sequence} problem, is a generalization of the TS problem, and a 
\textbf{PSPACE-complete} problem~\citep{Jerrum:1985}.
The diameter of the Cayley Graphs has been researched in the context of 
transposition trees~\citep{Akers:1989, Cooperman:1992,
Bafna:1998,Ganesan:Diameter:2012, Ganesan:Strictness:2012,Chitturi:2013,
Kraft:2015,Chitturi:2019}\textemdash i.e., transpositions generators from the 
edges of a tree.
It resulted in many heuristic algorithms to calculate upper-bounds that
do not depend on the vertex number of the Cayley Graph\footnote{In the case of
the TS, the vertex number of a Cayley Graph is $\worstCase{\vertexNumber!}$.}.

% TODO: Add one more example here

% Improve this paragraph
There is discussion about the relation of the TS problem and variants 
to parallel sorting on a SIMD machine consisting of several processors with
local memory connected by a network~\citep{Yamanaka:2015,Kawahara:2017}. 
Applications of the TS problem encompass a wide range of fields and some 
examples are: computing efficient interconnection network structures where the 
maximum delay could be measured by calculating the diameter of the 
network~\citep{Annexstein:1990}, computational biology~\citep
{Bafna:1998,Lenwood:2003}, model Wireless Sensor Networks (WSS)~\citep
{Wang:2007}, protection routing~\citep{Pai:2020} and qubit allocation for 
quantum computers~\citep{Siraichi:2018,Siraichi:2019}.
More examples can be found in \cite{Oswin:2021}.

The reconfiguration version of TS was first introduced in 2015~\citep{
Yamanaka:2015}, and further generalizations and variations were studied in the 
same year~\citep{Yamanaka:Colored:2015}.
The TS problem was first proved \textbf{NP-Complete}, and it remains 
even when restricted to bipartite graphs with degree bounded by 3. 
It is also \textbf{APX-complete} and \textbf{W[1]-hard} parameterized by number of swaps, 
but fixed parameter tractable (\textbf{FPT}) for the class of nowhere dense
graphs~\citep{Kawahara:2017,Miltzow:2016}.
Subsequently, it was proved that the problem remains hard even when both the 
treewidth and the diameter of the input graph are constant \citep{Bonnet:2018}.
There are some special classes of graphs that can be solved through a
exact polynomial time algorithm, almost all of them with good references 
listed in historical order by a study of TS in trees performed by 
\citep{Ahmad:2019}: cliques, paths, cycles, stars, brooms, 
lollipop~\citep{Kawahara:2017}, complete bipartite graphs and complete split 
graphs.
Concerning square of paths, there is a $2$-approximation algorithm 
\citep{Lenwood:2003}.
For trees, two $2$-approximation algorithms \citep{Yamanaka:2015,Miltzow:2016} 
are known and proven to be tight \citep{Ahmad:2019}.
It was showed that these known techniques for approximating TS on trees
prevent approximation factors less than 2~\citep{Oswin:2021}.
Both approximation algorithms can be adapted to general graphs, providing 
$\alpha$- and $4$-approximation algorithms, respectively, where $\alpha$ is the 
value of an $\alpha$-spanner tree of the input graph.
\cite{Bonnet:2018} conjectured that TS remains \textbf{NP-Complete}
even in trees and \cite{Ahmad:2019} reinforced this conjecture by showing a 
counterexample of the \textit{Happy Leaf Conjecture}~\citep{Vaughan:1991}.
And finally, \cite{Oswin:2021} showed that TS and other two variations (Weighted 
Token Swap and Parallel Token Swap) are \textbf{NP-Complete} on trees.

\section{Organization of the Work}
\label{sec:intro:organi}

This chapter focuses on introducing the mathematical tools and other 
already researched graph classes necessary for understanding the rest of this 
dissertation.
Chapter~\ref{chp:cographs} explains the main subject of this dissertation,
introducing the class of cographs.
The first section, Section~\ref{sec:solvindpermcycles}, presents the method 
for finding an optimal swap sequence for threshold graphs and the respective proof
of correctness, while Section~\ref{sec:deppermcycles} improves on the past 
proof to generalize the method for cographs.

The following chapter, Chapter~\ref{chp:ilp_models}, presents two initial integer 
linear programming models for the Token Swap problem and Parallel Token
Swap problem, being Section~\ref{sec:formulation_ts} and Section~\ref{sec:formulation_pts},
respectively.
Then, the subsequent sections and subsections render a discussion about each of 
the constraints and their design.
The dissertation is then concluded with a simple revision and exploration of 
what can be done in the future.

\section{Preliminaries}
\label{sec:intro:prelim}

For an integer $k$, the notation $\enum{k} \defeq \createSet{1, 2, 
\ldots, k}$ is used; and for a set $\vertexSet$, a mapping function $\defineFunc
{\mapFunc}{\vertexSet}{\vertexSet}$ is a bijective function that internally maps
elements of the set.
An identity map is a special mapping function $\idMap$ that maps every element
to itself. 
For a set $\vertexSet$, an ordering of the elements of the set is a bijective 
function $\defineFunc{\vertexOrder}{\vertexSet}{\mathbb{N}}$ and the 
shorthand $\vertexu \lessVOrder \vertexv$ for the comparison $\applyFunc{
\vertexOrder}{\vertexu} < \applyFunc{\vertexOrder}{\vertexv}$ of the order of 
two elements $\vertexu,\vertexv\in\vertexSet$ is adopted.
A graph $\graphG \defeq (\vertexSet, \edgeSet)$ is a pair of a vertex set 
$\vertexSet = \createSet{v_1, v_2, \ldots, 
v_{\vertexNumber}}$ and edge set 
$\edgeSet \subseteq {\vertexSet^2}$.
% The graph is called \textit{undirected} if and only if each edge $\createOrd
% {\vertexu, \vertexv} \in \edgeSet$ have the inverse $\createOrd{\vertexv, 
% \vertexu} \in \edgeSet$ and is called \textit{directed} otherwise.
The graph is called \textit{undirected} if and only if the set of edges 
$\edgeSet$ have unordered tuples, while for \textit{directed} graphs the
set of edges have ordered tuples.
The shorthand `$\vertexuv$' is used to describe a pair $\createOrd{\vertexu,
\vertexv}$ and the functions $\vertexFuncG$ and $\edgeFuncG$ are used to 
respectively retrieve the sets $\vertexSet$ and $\edgeSet$ when they are
omitted in the graph definition.
A \textit{subgraph} $\sub{\graphG} \defeq \createOrd{\sub{\vertexSet}, 
\sub{\edgeSet}}$ of the graph $\graphG$, denoted as $\sub{\graphG} \subgraph 
\graphG$, is a graph such that $\sub{\vertexSet} \subseteq \vertexFuncG$ and
$\sub{\edgeSet} \subseteq \edgeFuncG \cap \sub{\vertexSet}^{2}$ and is called 
a \textit{induced} subgraph when $\sub{\edgeSet} = \edgeFuncG \cap \sub
{\vertexSet}^{2}$.
The outdegree $\odeg{\vertexv}$ and indegree $\ideg{\vertexv}$ of a vertex 
$\vertexv \in \vertexFuncG$ is the number of edges $\createOrd{\vertexv, 
\vertexw}$, $\forall\vertexw \in \vertexFuncG$ and $\createOrd{\vertexw, 
\vertexv}$, $\forall\vertexw \in \vertexFuncG$ that belong to $\edgeFuncG$, 
respectively.
The degree $\tdeg{\vertexv}$ of a vertex is defined as the sum of 
$\ideg{\vertexv}$ and $\odeg{\vertexv}$.
The open neighborhood $\openNeigh{\graphG}{\vertexu}$ of a vertex is defined 
as a set of all vertices $\vertexv$ such that $\vertexuv \in \edgeFuncG$ and
$\vertexu$ itself is included.
If $\vertexu$ is not included, the set is called a closed neighborhood as is
denoted as $\closedNeigh{\graphG}{\vertexu}$.
The complement of a graph $\graphG$ is a graph $\sub{\graphG}$ with the same
vertice set $\vertexFuncG$ such that an edge exists between two distinct 
vertices if and only if they were not adjacent on $\graphG$.

\addExample{composition.tex}

An instance of the TS problem is composed of a graph $\graphG \defeq \createOrd
{\vertexSet,\edgeSet}$ and a mapping function $\initMap$ that denotes an initial 
token placement on vertices of $\graphG$.
A \textit{swap} is defined by a vertex pair $\swaps \defeq \createOrd{\vertexu,
\vertexv}$ and it can be \textit{applied} to exchange two tokens in the current 
mapping of an instance provided that $\createOrd{\vertexu,
\vertexv} \in \edgeSet$.
More formally, any swap $\swaps = \createOrd{\vertexu,\vertexv}$ can be 
represented as a mapping function $\mapFunc_s$, such that $\applyFunc
{\mapFunc_s}{u} = v$, $\applyFunc{\mapFunc_s}{v} = u$ and $\applyFunc
{\mapFunc_s}{w} = w$, $\forall w \in \vertexSet\setminus \createSet{u,v}$, and 
applied to any token placement $\mapFunc$ by the composition $\mapFunc\groupOp
\mapFunc_s = \mapFunc_1$.
An example on how the composition operation can swap function values is shown
in Figure~\ref{img:applyswap}.

Let $\graphG \defeq \graphPair$ be a graph.
It can be said that two vertices $\vertexu, \vertexv \in \vertexSet$ are 
\textit{connected} if there is at least one \textit{path of vertices} 
$\createOrd{\vertexv_1, \vertexv_2, \ldots, \vertexv_k}$ in $\graphG$ such that 
$\vertexv_{i}\vertexv_{i+1} \in \edgeSet, \forall_{i\in\enum{k-1}}$, 
$\vertexv_{1} = \vertexu$ and $\vertexv_{k} = \vertexv$.
A graph $\graphG$ is connected if all vertex pairs of $\vertexSet$ are 
connected and disconnected otherwise.
The function $\distG{\vertexu}{\vertexv}$ denotes the minimum number of edges in 
any path of vertices between nodes $\vertexu$ and $\vertexv$ for a graph 
$\graphG$.
Note that this distance is considered infinite in the case there is no 
path between the two vertices.
A token placement $\mapFunc$ is considered \textit{valid} if all vertex pairs
$\createOrd{\applyFunc{\mapFunc}{\vertexu}, \vertexu}$ are connected.
As all token placements in this paper are assumed to be valid, every 
disconnected graph of a TS instance can be separated into smaller connected 
instances with restrictions to the domain of the initial placement.
Hence, all graphs can be assumed connected without loss of generality.

Let $\swapSeq \defeq \createOrd{\swaps_1, \swaps_2, \ldots, \swaps_k}$ be a 
sequence of swaps.
A swap sequence $\swapSeq$ \textit{solves} an instance of the TS problem if and 
only if the identity function is resulted by applying each swap iteratively, as 
shown in Equation~\ref{equ:seqid}. 
Every intermediate mapping function created is represented by a $\mapFunc_p$, 
where $p$ is the number of swaps applied from the initial configuration.
Moreover, two placement mappings are said to be \textit{adjacent} if there is 
exactly one swap to transform one placement into another.
The Token Swap problem asks for a sequence of swaps that solves a given 
instance with the minimal number of swaps $k$.

\addEquation{mappings.tex}

It is important to note that this version of the problem is equivalent to the
problem of finding a swap sequence between two arbitrary mapping functions
$\initMap$, $\mapFunc_e$ by the use of the following process of \textit{token
renaming} from $\initMap$ to $\sub{\initMap}$: if a token $\tokeni$ has
$\applyFunc{\initMap} {\tokeni} = \vertexu$, the $\tokeni$ is renamed to the 
token $\applyFunc{\mapFunc_{e}}{\vertexu}$, generating an instance of the TS
problem with the same graph and initial mapping $\sub{\initMap} = \mapFunc_{e}
\groupOp \initMap$. 
Given a $\swapSeq$ that solves the TS instance, the vertices can just be renamed 
back to get a swap sequence of the initial problem.
Any swap sequence $\swapSeq$ is also \textit{reversible}, meaning that if 
$\swapSeq$ transforms $\mapFunc_{i}$ to $\mapFunc_{j}$, then it is possible
to transform $\mapFunc_{j}$ to $\mapFunc_{i}$ by applying the swap 
sequence of $\swapSeq$ in the reverse order.
This property is specially useful in some applications like quantum computing,
where all quantum logic operations must be reversible.

% Add an LCA example

Another important definition needed throughout this work is the notion of the
\textit{Lowest Common Ancestor} ($LCA$), sometimes called nearest common 
ancestor, for trees. 
A \textit{tree} $\graphG$ is a undirected and connected graph that has no cycles 
and is called \textit{rooted}, denoted as $\graphrG$, if there exists a special 
node $r \in \vertexFuncG$ called \textit{root} that functions as a reference 
node for heights in the graph.
Any vertex $\vertexu$ of a tree with $\tdeg{\vertexu} = 1$ is called a leaf, 
with the exception of the root of a rooted tree.
A \textit{subtree} is a subgraph of a tree that is still a tree. 
A subtree $\sub{\graphG}$ of a rooted tree $\graphrG$ can also rooted in relation 
to the original root, as the nearest vertex from the subtree to $r$ in 
$\graphrG$ will be the subtree's root.
For a given rooted tree $\graphrG$, $\vertexu,\vertexv \in \vertexFuncrG$, the 
lowest common ancestor between $\vertexu$ and $\vertexv$, denoted as $\lca
{\vertexu, \vertexv}$, is the lowest node such that both nodes are descendants 
of.
In another words, it is the nearest shared ancestor of both $\vertexu$ and
$\vertexv$.

% Add an image here as a example of the process of subset LCA.

This notion is extended to calculate the $LCA$ for any vertex subset of the 
graph. 
Let $\graphrG$ be a rooted tree and $\sub{\vertexSet} \subseteq \vertexFuncrG$ a 
vertex subset. 
The lowest common ancestor $\lca{\sub{\vertexSet}}$ is the nearest node from 
the root in the set of nodes built from the lowest common ancestors between 
every pair of nodes in $\sub{\vertexSet}$.
% The nearest node from the set of farthest nodes given by the pairwise $LCA$ is
% the node that is the lowest common ancestor of the set of nodes $\sub
% {\vertexSet}$.

Given the set of every pairwise $LCA$ of the subset, the $LCA$ of the entire 
subset $\sub{\vertexSet}$ is the nearest node from the root in the set.
The Theorem~\ref{thr:lca_subset} helps in understanding that the process indeed 
generates the lowest common ancestor of the subset.

\begin{theorem}
\label{thr:lca_subset}
Let $\graphrG$ be a rooted tree and a subset $\sub{\vertexSet} \subseteq 
\vertexFuncrG$. 
The $\lca{\sub{\vertexSet}}$ gives us the lowest common ancestor of the vertex 
subset.
\end{theorem}

\begin{proof}
Let $J$ be the set of all lowest common ancestors resulting from the pairwise
node calculation, $\vertexw$ be the node of $J$ located the nearest from
the root and $\vertexu$,$\vertexv$ the nodes of $\vertexFuncrG$ such that
$\lca{\vertexu,\vertexv} = \vertexw$.
To show that this node is indeed a common ancestor of all nodes in 
$\sub{\vertexSet}$, suppose, by contradiction, that there is a node $p \in 
\vertexFuncrG$ that $\vertexw$ is not a ancestor of.
Then, if we form a pair from $\vertexu$ or $\vertexv$ and $p$, the distance of 
the lowest common ancestor $\lca{\vertexu, p}$ or $\lca{\vertexv, p}$ 
\textit{must} be lower than the distance of $\vertexw$, otherwise $\vertexw$ 
would be a common ancestor, as this node will have to exist in the subtree where 
$\vertexw$ is the root.
But this create a contradiction, as $\vertexw$ were the nearer node from the 
root from the set of pairwise lowest common ancestors.

To prove that this node is the \textit{lowest} common ancestor, we assume, by
contradiction, that there is another node from $J$ that is lower and is a common
ancestor to all nodes in $\vertexSet'$.
Then, $\vertexw$ is not the lowest common ancestor of nodes $\vertexu$ and 
$\vertexv$, creating a contradiction.
\end{proof}

% Add LCA example here

The problem of calculating $\lca{\vertexu, \vertexv}$ is called \textit{offline} 
when the graph and queries are being given as input and has been first proposed 
in \citep{Aho:1973} with an optimally efficient algorithm. 
Other versions were studied later~\citep{Harel:1984}, with many different 
algorithms and general improvements over the years~\citep{Alstrup:2004}.
Some of these algorithms present a linear time pre-process stage that creates a 
data structure that can be dynamically queried in asymptotically constant time 
and others offers efficient algorithms that queries on trees that can be changed
dynamically.
The exact improvements and methods used to calculate the lowest common ancestor 
go out of the scope of this dissertation.
From the above, the calculation of the $\lca{\sub{\vertexSet}}$ can also be
derived in optimally efficient time, as the number of pairs of vertex is bounded 
by $\worstCase{\sizeof{\sub{\vertexSet}}^2}$.

A Conflict Graph $\conflictGraph_{\mapFunc} \defeq \createOrd{\vertexFuncG, 
\edgeSetConflict}$ is a directed graph that, for a token placement 
$\mapFunc$ of a graph $\graphG$, an edge $\createOrd{\vertexu, \vertexv} \in 
\edgeSetConflict$ if and only if $\applyFunc{\mapFunc}{\vertexu} = \vertexv$.
Note that each node has outdegre 1, as there can be only one token per vertex, 
and the directed graph may contain self-loops when a token is already in the
correct vertex.
In this graph, configurations can be characterized by a set of directed 
cycles $\applyFunc{\cycleSet}{\conflictGraph} = \createSet{\cycle_{0}, 
\cycle_{1}, \ldots, \cycle_{p}}$, for $p \in \mathbb{N}$, as seen in 
Lemma~\ref{lem:dircycle}.

\begin{lemma}
\label{lem:dircycle}
\citep{Yamanaka:2015}
Let $\conflictGraph_{\mapFunc}$ be a conflict graph of a graph $\graphG$.
Then, every component in $\conflictGraph$ is a directed cycle.
\end{lemma}

The \textit{joint representation} is an easy way to visually represent a Conflict 
Graph and the original graph as one, as seen in Figure~\ref{img:ts-example}, 
and will be used throughout this dissertation.
Lemma~\ref{lem:cgbound} and Lemma~\ref{lem:disjcycles} underline two basic 
properties of this set of cycles.
The cyclic representation $\cycle = \createCyc{\vertexu_{1}, \ldots, \vertexu_
{k}}$ denotes a cycle such that $\vertexu_{i}\vertexu_{i+1} \in \edgeFunc
{\cycle}$, $\forall i \in \enum{k-1}$ and $\vertexu_{k}\vertexu_{1} \in 
\edgeFunc{\cycle}$.

\begin{lemma}
\label{lem:disjcycles}
Let $\conflictGraph_{\mapFunc}$ be a conflict graph of a graph $\graphG$.
For each pair of cycles $\cycle_{i},\cycle_{j} \in \applyFunc{\cycleSet}
{\conflictGraph}$, $\vertexFunc{\cycle_{i}} \cap \vertexFunc{\cycle_{j}} =
\createSet{\emptyset}$.
\end{lemma}

\begin{proof}
By definition, as each vertex can have only one token positioned in it and 
each token can have only one target vertex.
\end{proof}

\begin{lemma}
\label{lem:cgbound}
Let $\conflictGraph_{\mapFunc}$ be a conflict graph of a graph $\graphG$.
The number of cycles in $\conflictGraph_{\mapFunc}$ is bounded by 
$\worstCase{\sizeof{\vertexFuncG}}$ and is exactly $\sizeof{\vertexFuncG}$ only 
when the current configuration is the identity map $\idMap$.
\end{lemma}

\begin{proof}
By Lemma~\ref{lem:disjcycles}, the maximum number of cycles happens when every 
vertex is part of a distinct disjoint cycle, resulting in one self-loop for 
each.
\end{proof}

Let $\mapFunc$ be a token configuration and $\conflictGraph_{
\mapFunc_{i}}$ the related conflict graph with permutation cycle set 
$\applyFunc{\cycleSet}{\conflictGraph_{\mapFunc_{i}}}$.
Every swap that can applied to $\mapFunc_{i}$ transform $\cycleSet$ in some way.
These transformation can be classified in two types: \textit{Merge} and 
\textit{Split}.
They are described in the following paragraph.

Take two distinct permutation cycles $\cycle_{i},\cycle_{j} \in \applyFunc
{\cycleSet}{\conflictGraph_{\mapFunc_{i}}}$ and assume, without loss of 
generality, that $\cycle_{i} = \createCyc{\vertexu, \vertexu_{1}, \ldots, 
\vertexu_{k}}$ and $\cycle_{j} = \createCyc{\vertexv, \vertexv_{1}, \ldots, 
\vertexv_{p}}$, for $k,p \in  \mathbb{N}_{0}$.
A Merge is a swap $\createOrd{\vertexu, \vertexv}$ that is applied between the
two cycles $\cycle_{i}$ and $\cycle_{j}$ resulting in a configuration 
$\sub{\mapFunc_{i}}$ such that $\applyFunc{\cycleSet}{\conflictGraph_{\sub{\mapFunc_{i}}
}} = (\applyFunc{\cycleSet}{\conflictGraph_{\mapFunc_{i}}} \setminus \createSet
{\cycle_{i}, \cycle_{j}}) \cup \createSet{\cycle_{ij}}$, where $\cycle_{ij} = \createCyc
{\vertexv,\vertexu_{1},\ldots,\vertexu_{k}, \vertexu, \vertexv_{1}, \ldots, 
\vertexv_{p}}$.
Now, take a permutation cycle $\cycle \in \applyFunc{\cycleSet}{\conflictGraph_
{\mapFunc_{i}}}$ and assume, without loss of generality, that $\cycle = 
\createCyc{\vertexu,\vertexu_{1},\ldots,\vertexu_{k}, \vertexv, \vertexv_{k+1}, 
\ldots, \vertexv_{k+p}}$, for $k,p \in \mathbb{N}_{0}$.
A Split is a swap $\createOrd{\vertexu, \vertexv}$ that is applied in one cycle 
$\cycle$ resulting in a configuration $\sub{\mapFunc_{i}}$ such that $\applyFunc
{\cycleSet}{\conflictGraph_{\sub{\mapFunc_{i}}}} = (\applyFunc{\cycleSet}{
\conflictGraph_{\mapFunc_{i}}} \setminus \createSet{\cycle}) \cup \createSet{\cycle_{i}, 
\cycle_{j}}$, where $\cycle_{i} = \createCyc{\vertexv, \vertexu_{1}, \ldots, 
\vertexu_{k}}$ and $\cycle_{j} = \createCyc{\vertexu, \vertexv_{k+1}, \ldots, 
\vertexv_{k+p}}$.

\begin{lemma}
\label{lem:allswaps}
Let $\graphG$ be a graph and $\mapFunc$ be a configuration of a TS problem 
instance.
Any possible swap $\createOrd{\vertexu, \vertexv} \in \edgeFuncG$ is either a 
merge or a split transformation in $\applyFunc{\cycleSet}{\conflictGraph_
{\mapFunc}}$.
\end{lemma}

\begin{proof}
By the conflict graph definition, every vertex must have outdegree one and 
indegree one.
By the Lemma~\ref{lem:dircycle}, these vertices can only be arranged in loops 
or self-loops.
As every vertex is on a loop and vertices between cycles are disjoint, as shown
in Lemma~\ref{lem:disjcycles}, every swap must be applied either internally on
a cycle or between two cycles.
A swap applied internally on a cycle is called a split swap and a swap applied
between two cycles is called a merge swap by the above definition.
\end{proof}

From these transformations, there are two special cases that are worth 
mentioning: Swap between a cycle of size one and any other cycle and swap on an 
edge of the cycle.
Figure~\ref{img:mergesplit} is an example to help visualize split and merge swaps.
These two cases can be used as tools to add or remove, respectively, a node from 
a cycle and will be useful in this dissertation.
For any cycle $\cycle \in \cycleSet$ and vertex $\vertexv \in \vertexFunc{\graphG}$,
it is said that $\vertexv$ \textit{dominates} the cycle $\cycle$ if and only if
the vertices $\vertexFunc{\cycle}$ are a subset of the open neighborhood of 
$\vertexv$, $\vertexFunc{\cycle} \subseteq \openNeigh{\graphG}{\vertexv}$.

\addExample{mergesplit.tex}

%--- ADD >

Let $\mapFunc$ be a token placement map of a TS instance.
The sum of the distances $\graphDist{\applyFunc{\mapFunc}{\vertexu}}{\vertexu}$
is the sum of the distances between each token to its target vertex. 
With the sum, one could test if a swap sequence $\swapSeq$ solves the instance
by checking if is 0 for the resulting placement $\mapFunc_{\sizeof{\swapSeq}}$.
For trees, every swap can be classified in one of three categories related to 
the sum of distances of an instance: (a) The swap decreases the sum by two 
through moving two tokens closer to its target vertices, also called a 
\textit{happy swap}; (b) the swap does not change the total sum by moving one 
token closer and one token further from their target vertices and (c) the swap 
increases the sum by two, as it moves two tokens further from its target 
vertices.

Intuitively, one could think that any swap sequence that solves a TS instance 
with swaps restricted to categories (a) and (b) will have less swaps than a 
swap sequence that solves the same instance and uses swaps of category (c) 
\textemdash as \citep{Smith:1999} tried to prove, but subsequently found an 
error~\citep{Smith:2011:Corrigendum}.
Then, \citep{Vaughan:1991} conjectured that any optimal swap sequence would 
not swap already correct tokens on leafs and called them \textit{happy leafs}, 
resulting in the so-called \textit{Happy Leaf Conjecture}\footnote{The original 
paper actually claimed a stronger conjecture with the Happy Leaf Conjecture as 
a special case.}.
This conjecture was disproved by \citep{Ahmad:2019}, as they shown that there is 
a class of infinite trees that necessarily need (c) swaps to achieve the optimal 
number of swaps and that any algorithm that do not consider swaps on 
happy leafs has an approximation factor of \textit{at least} $\frac{4}{3}$ for 
general graphs and trees.
The two approximative algorithms cited in this chapter can only generate swap 
sequences of category (a) and (b).

%--- ADD <

In the realm of Group Theory, it is possible to represent each permutation cycle 
$\cycle$ as a mapping function $\mapFunc_{\cycle}$, as seen before in Figure~\ref{img:applyswap}, 
by a decomposition of a original configuration $\mapFunc$.
A \textit{partition} $\partitionFunc{\setA} = \createSet{\partitionElem_{1}, 
\ldots, \partitionElem_{k}}$ of a set $\setA$ is a grouping of its elements into 
non-empty subsets $\partitionElem_{i} \in \enum{k}$, such that every element of 
the set $\setA$ belong to exactly one of these subsets.
Let $\graphG$ be a graph, $\mapFunc$ be a token configuration and 
$\partitionFunc{\vertexFuncG}$ be a vertex partition such that every 
configuration $\mapFunc_{\partitionElem}$, $\partitionElem \in 
\partitionFunc{\vertexFuncG}$ is a valid token configuration on $\graphG$.
This partitioning is called a \textit{valid} partition.

\addExample{partition.tex}

From the configuration $\mapFunc$ and valid element of the partition $\partitionElem 
\in \partitionFunc{\vertexFuncG}$, a configuration $\mapFunc_{\partitionElem}$ is 
built such that $\applyFunc{\mapFunc_{\partitionElem}}{\vertexv} = \applyFunc
{\mapFunc}{\vertexv}$, $\forall \vertexv \in \partitionElem$ and $\applyFunc
{\mapFunc_{\partitionElem}}{\vertexv} = \vertexv$, $\forall \vertexv \in 
\vertexFuncG \setminus \partitionElem$.
The $\textit{coarsest}$ valid partition is defined as the partition where for 
each other partition of greater size there is at least one subset 
$\partitionElem$ that the configuration $\mapFunc_{\partitionElem}$ do not 
result in a valid token configuration.
For any valid partition $\partitionFunc{\vertexFuncG}$ of a TS problem instance, 
the composition of all element configurations $\mapFunc_{\partitionElem}$ 
results in the original partition $\mapFunc$ as seen in Equation~\ref
{equ:partcomp}, with $k = \sizeof{\partitionFunc{\vertexFuncG}}$.

\addEquation{partcomp.tex}

\begin{lemma}
\label{lem:cycmap}
Let $\partitionFunc{\vertexFuncG}$ be a coarsest valid partition of a graph 
$\graphG$ and a configuration $\mapFunc$ of a TS instance.
There is a partition configuration $\mapFunc_{\partitionElem}$, $\partitionElem 
\in \partitionFunc{\vertexFuncG}$ if and only if there is a cycle $\cycle \in 
\applyFunc{\cycleSet}{\conflictGraph_{\mapFunc}}$ such that $\vertexFunc{\cycle} 
= \partitionElem$ and $\applyFunc{\mapFunc_{\partitionElem}}{\vertexv} = 
\vertexu$ if and only if $\createOrd{\vertexv, \vertexu} \in \edgeFunc{\cycle}$.
\end{lemma}

\begin{proof}
$\createOrd{\rightarrow}$ Assume any partition $\partitionElem \in 
\partitionFunc{\vertexFuncG}$ with configuration $\mapFunc_{\partitionElem}$. 
As this configuration is valid by the definition, there is a 
set of cycles $\applyFunc{\cycleSet}{\conflictGraph_{\mapFunc_
{\partitionElem}}}$ and every vertex in
$\vertexFuncG \setminus \partitionElem$ must be in a self-loop of $\applyFunc
{\cycleSet}{\conflictGraph_{\mapFunc_{\partitionElem}}}$.
If the subgraph of the vertices of $\partitionElem$ on $\conflictGraph_{
\mapFunc_{\partitionElem}}$ is disconneted, then each connected subgraph of this
subgraph is a disjoint cycle and it is possible to create a smaller partitions 
of $\partitionElem$ that are still valid, which is a contradiction on the 
coarsest valid partition.
So, there is exactly one $\cycle \in \applyFunc{\cycleSet}{\conflictGraph_
{\mapFunc_{\partitionElem}}}$, such that $\vertexFunc{\cycle} = \partitionElem$.
From the definition of the configuration $\mapFunc_{\partitionElem}$ and the
original configuration $\mapFunc$, where $\applyFunc{\mapFunc}{\vertexv} = 
\vertexu$, $\applyFunc{\mapFunc_{\partitionElem}}{\vertexv} = \applyFunc
{\mapFunc}{\vertexv} = \vertexu$ for $\createOrd{\vertexv, \vertexu} \in 
\edgeFunc{\conflictGraph_{\mapFunc}} \cap \partitionElem^{2} = \edgeFunc
{\cycle}$. 

$\createOrd{\leftarrow}$ Assume a cycle $\cycle \in \applyFunc{\cycleSet}
{\conflictGraph_{\mapFunc}}$. 
A valid partition of the vertices $\partitionElem \in \partitionFunc
{\vertexFuncG}$ can be built from the vertices of each cycle $\vertexFunc
{\cycle} = \partitionElem$, for each $\cycle \in \applyFunc{\cycleSet}
{\conflictGraph_{\mapFunc}}$.
If this created partition is not the coarsest, then there is a cycle in 
$\applyFunc{\cycleSet}{\conflictGraph_{\mapFunc}}$ that is not a cycle, but a
disjoint union of two or more cycles, which is absurd.
By definition, for each $\createOrd{\vertexv, \vertexu} \in \edgeFunc{\cycle}$, 
the related partition configuration will be $\applyFunc{\mapFunc_
{\partitionElem}}{\vertexv} = \vertexu$ and $\applyFunc{\mapFunc_
{\partitionElem}}{\vertexv} = \vertexv$ for $\vertexv \in \vertexFuncG \setminus 
\vertexFunc{\cycle}$.
\end{proof}

Note that for Equation~\ref{equ:partcomp} the order of the applications
between the functions does not matter.
Lemma~\ref{lem:cycmap} proves that this partition representation of a 
token configuration is equivalent to the cycle set representation.
Partitioning swaps can be useful to determine which swaps can be run
in parallel in variants of the token swap like Parallel Token Swap.
The Corollary~\ref{cor:sametokens} shows an interesting interaction
between tokens and swaps that can be used to optimize swap sequences.

\begin{corollary}
\label{cor:sametokens}
Let $\swapSeq$ be an optimal sequence of swaps for a graph $\graphG$ and
initial configuration $\initMap$.
Then, for any two distinct swaps $\swaps_{1},\swaps_{2} \in \swapSeq$,
they cannot interact with the same two tokens.
\end{corollary}

\begin{proof}
Suppose that there is two swaps $\swaps_{1},\swaps_{2} \in \swapSeq$ that 
exchange the same two tokens $\tokeni$ and $\tokenj$ and assume that $\swaps_{1}$
appears before on the sequence than $\swaps_{2}$.
Now, take every swap of $\swapSeq$ (in-order) that are in-between $\swaps_{1}$ 
and $\swaps_{2}$ that exchange either $\tokeni$ or $\tokenj$ and call this
new sequence $\swapSeq_{\tokeni,\tokenj}$.
Notice that, by removing the swap $\swaps_{1}$ from $\swapSeq$, every swap
on $\swapSeq_{\tokeni,\tokenj}$ that interacts with only $\tokeni$ will now
be interacting with $\tokenj$ and vice-versa.
Swaps that interact with both tokens will continue interacting with both, 
but the position of the tokens in the vertices will be swapped.

Removing this swap does not affect any other token, as it doesn't matter
which token is paired with the other tokens in the swaps.
As the tokens on $\swaps_{2}$ will now be already swapped, it is possible
to just remove this swap.
This second removal guarantees that the tokens $\tokeni$ and $\tokenj$ will
be in the same position as the original sequence $\swapSeq$, creating a new
$\sub{\swapSeq}$ that can solve the instance and have less swaps than
$\swapSeq$, which is absurd, as $\swapSeq$ is optimal.
\end{proof}

\section{Graph Classes and Upper Bounds}
\label{sec:intro:classes_uppers}

This section is a summary of results regarding  Token Swapping
on specific graph classes and upper bounds on the numbers of swaps.
Special focus will be given to graph classes that are related to
cographs, as this class is the main focus of this dissertation.

\addExample{classes.tex}

The problem of swapping tokens on paths, with an example in Figure~\ref
{img:classes:path}, goes back to the classic problem of ordering an array of 
integers by the exchange of adjacent numbers, which an efficient algorithm is 
known for quite some time~\citep{Jerrum:1985,Knuth:1998}.
This algorithm, detailed by Algorithm~\ref{alg:path}, is a variation of the 
well known bubble-sort algorithm and the resulting number of swaps is well
defined by the \textit{inversion number}.
Let $\initMap$ be a mapping function for a path $\graphG$ and $\tokeni,\tokenj 
\in \applyFunc{\vertexSet}{\graphG}$. 
If $\tokeni \lessVOrder \tokenj$ and $\applyFunc{\initMap}{\tokeni} \moreVOrder 
\applyFunc{\initMap}{\tokenj}$, then either the elements $(\tokeni, \tokenj)$ or 
the places $(\applyFunc{\initMap}{\tokeni}, \applyFunc{\initMap}{\tokenj})$ are
called an inversion of $\initMap$.
The size of the set of all inversions of $\initMap$ is called the inversion 
number.
For any initial mapping function $\initMap$, the minimum number of swaps is
upper bounded by the maximum number of inversions (and thus, the diameter
of the corresponding Cayley Graph is also bounded by this same value), which 
is $\worstCase{\vertexNumber^2}$.

\addAlgorithm{path.tex}

% Cycles

For cycle graphs, shown in Figure~\ref{img:classes:cycle}, the method is an 
extension of the inversion number method of a path because the inversion number 
cannot be applied directly, as token can be moved either in a clockwise or 
anticlockwise manner to its target vertice.
The optimal algorithm, presented by \cite{Jerrum:1985}, first finds a feasible
solution through an integer program and proves that this program can be calculated in
polinomial time by restricting the search space to optimal solutions and then 
applying transformations to contract the resulting swap sequence until no other 
transformation is possible, resulting in a optimal swap sequence.

% Brooms
% Lollipops
A broom graph is a graph such that the center of a star is connected to a path
and a lollipop is a graph where a node from a cycle is connected to a path and
are respectively shown in Figure~\ref{img:classes:broom} and Figure~\ref
{img:classes:lollipop}. 
The proofs of correctness of the polynomial algorithms for brooms and lollipop 
graphs are similar and based on a evaluation function on the sum of the distances 
of each token to it's target vertice.
Each swap alter this evaluation function by adding or subtracting the distances 
and the identity function is achieved if and only if the sum is zero.
The polynomial algorithms are fairly simple and are presented by 
\cite{Kawahara:2017}.

\subsection{Token Swapping on Stars}
\label{sec:intro:classes_uppers:star}

A star, shown in Figure~\ref{img:classes:star}, is a tree such that every vertice
is a leaf except by the \textit{center vertice} that is connected to every leaf 
vertice.
In this graph, as cycles are disjoint, there is exactly one cycle such that
the center vertice is a member of and can be solved in exactly the number of leafs
in the cycle.
For every other cycle, the minimum number of swaps to solve is the number of
leafs in the cycle plus one, as it is necessary to move the center token to
allow other tokens to achieve their destination.
Figure~\ref{img:starswaps} is an example of finding an optimal swap sequence on 
a star graph instance.

\addExample{starswaps.tex}

Based on this, the total number of swaps is given by $y + l$, where $y$ is the
total number of incorrect leafs and $l$ is the number of cycles in the configuration
that have size greater than 2 and the center vertex is not a part of~\citep
{Akers:1989,Ahmad:2019}.
Note that, in this case, each cycle is solved separately and no interaction
\textit{between} the initial cycles is needed to achieve the identity configuration,
also meaning that there is no specific order of swaps between cycles, as long
the cycle with the center vertex is solved first or the center vertex is
assumed correct for the other cycles.
These past results also give an upper-bound on the diameter of the corresponding 
Cayley Graph.

\subsection{Token Swapping on Cliques}
\label{sec:intro:classes_uppers:clique}

Clique graphs, also called complete graphs, are graphs such that there is an 
edge connecting every pair of distinct vertices.
The Figure~\ref{img:classes:clique} is an example of a complete graph with
five nodes.
In a clique graph with a token configuration, every token is either already 
correct or adjacent to its target vertex.

Then, for each cycle of size $k$, it is possible to solve it using exactly
$k-1$ swaps, as every swap can move a token to its target vertex, except for
the swap in a cycle of size 2 that solves two tokens at the same time.
The total swaps needed for any clique is then $n - r$, where $n$ is the total of
vertices and $r$ is the total of cycles and the upper bound $n - 1$ is given when
there is exactly one cycle on the configuration~\citep{Kim:2016}.

\subsection{Token Swapping on Complete Split}
\label{sec:intro:classes_uppers:compsplit}

A split graph is a graph such that its vertex set can be partitioned into
a clique and an independent set, a set of vertices with no edges between any
pair of vertices of the set.
The Figure~\ref{img:classes:compsplit} is an example of a split graph with six
nodes: A clique of size four and an independent set of size two.
A complete split graph has every possible edge between the vertices of the
clique and the independent set.
For every cycle of a token swap instance on a complete split graph, the cycle
must be either completely contained on the clique, the independent set or
having vertices from both structures.

Let $k$ be the size of a cycle in a Token Swap instance  on a complete split
graph.
If the cycle is completely contained on the clique, the cycle can be independently
solved as a normal clique cycle with $k -1$ swaps.
However, cycles contained in the independent set have no edges between them
and every token must have to be moved to the clique to be solved.
The process guarantees a $k + 1$ sequence of swaps to solve the cycle.
For cycles contained in both structures, it is possible to use a clique vertex
to pivot every swap through it, similarly as solving a cycle that contains the
center vertex of a star, in $k - 1$ swaps.
The entire process is outlined in \cite{Yasui:2015}.

\subsection{Token Swapping on Complete Bipartite}

A complete bipartite graph, shown in Figure~\ref{img:classes:compbip}, 
is a graph such that its vertex set can be partitioned into two independent sets
with all edges between them.
Similarly to complete split graphs, for an instance of token swap on a complete
bipartite graph with partitions $X$ and $Y$, the cycles can be divided into: 
a) It is completely contained in partition $X$; b) It is completely contained
in partition $Y$; c) It has vertices from partitions $X$ and $Y$; and d) Is a
cycle of size one.

Cycles of type c) can be independently solved in $k - 1$ swaps and cycles of
type d) are already correct.
Although cycles of type a) and b) are contained in separate independent sets
and can be solved efficiently in the way showed by Section~\ref{sec:intro:classes_uppers:compsplit} 
in $k + 1$ swaps each, it is possible to save swaps by \textit{merging} a cycle
of type a) and b)~\citep{Yamanaka:2015}.
This notion of merging cycles to save swaps will be generalized into cographs,
which form a superclass of complete bipartite graphs, in Section~\ref{sec:deppermcycles}.

\chapter{Solving Token Swap on Cographs}
\label{chp:cographs}

A cograph is defined recursively as follows:

\begin{itemize}
    \item A graph with a single vertex is a cograph;
    \item If $\graphG_{1}, \graphG_{2}, ... ,\graphG_{k}$ are cographs, then 
    so is their disjoint union;
    \item if $\graphG$ is a cograph, then so is its complement $\overline{G}$.
\end{itemize}

A cotree $\cotreeOf{\graphG}$ of a cograph $\graphG = \graphPair$ is a rooted 
tree representing its structure. 
The leaves of $\cotreeOf{\graphG}$ are exactly $\vertexSet$ and each internal 
node is labelled 0 or 1 and will be called 0-node and 1-node, respectively.
The children of an 1-node are 0-nodes or leaves and the children of a 0-node 
are 1-nodes or leaves. 
Two vertices are adjacent in a cograph if and only if their lowest common 
ancestor (LCA) in the cotree is an 1-node.
The cotree of any particular cograph is unique.

Note that some authors also describe another rule called the \textit{join} 
operation to build cographs.
Given two cographs $\graphG_{i}$ and $\graphG_{j}$, a join $join(\graphG_{i},
\graphG_{j})$ results in a graph $\sub{\graphG}$ such that $\vertexFunc{\sub
{\graphG}} = \vertexFunc{\graphG_{i}} \cup \vertexFunc{\graphG_{j}}$ and 
$\edgeFunc{\sub{\graphG}} = \edgeFunc{\graphG_{i}} \cup \edgeFunc{\graphG_{j}}
\cup ( \vertexFunc{\graphG_{i}} \times \vertexFunc{\graphG_{j}} )$.
This operation can be described by the basic cograph operations $join(\graphG_{i},
\graphG_{j}) = \compl{\compl{\graphG_{i}} \cup \compl{\graphG_{j}}}$, resulting 
in a valid operation to build cographs.
Figure~\ref{img:cograph} introduces a simple example of a cograph and its cotree
representation.
Note that in a cotree, 1-nodes represent joins and 0-nodes represent disjoint 
unions between the leaves of their subtrees.

By looking at most of the classes presented at Section~\ref{sec:intro:classes_uppers} 
that have a polinomial time algorithm, it is possible to notice that most of this classes 
(stars, cliques, complete bipartite and complete split graphs) are subclasses 
of the cograph class.
So, it seemed logical to begin inquiring into other subclasses of cographs for a more
definitive pattern that could be used for the more general class.
From this intuition, the class of threshold graphs were the first chosen as research
subject, as it is both a superclass of complete split and a subclass of the cographs.
From the results that were found emerged an interesting pattern that could be generalized
for the cograph class and will be presented in this chapter.

\addExample{cograph.tex}

Let $\graphG$ be a cograph and $\cotreeOf{\graphG}$ the respective cotree. 
Let $\initMap$ be a token placement on $\graphG$ and $\conflictGraph$ the related 
conflict graph with set of permutation cycles $\applyFunc{\cycleSet}{\conflictGraph}$.
% Let $\vertexu$ be any 0-node and $\vertexw$ be any 1-node of $\cotreeOf{\graphG}$. 
A $\zeroCycles$ set is the set of all cycles $\cycle \in \cycleSet$ such that
the lowest common ancestor in the cotree of $\graphG$ of the vertices of the cycle
is $\vertexu$, $\lcaOf{\cotreeOf{\graphG}}{\vertexFunc{\cycle}} = \vertexu$, with 
$\vertexu$ being a 0-node of the cotree.
A $\oneCycles$ set is the set of all cycles $\cycle \in \cycleSet$ such that
$\lcaOf{\cotreeOf{\graphG}}{\vertexFunc{\cycle}} = \vertexw$, with $\vertexw$ being any
1-node of the cotree, or $\lcaOf{\cotreeOf{\graphG}}{\vertexFunc{\cycle}} = \vertexv$ 
in the case that $\cycle = \createOrd{\vertexv}$.
These two sets are disjoint, their union is exactly the cycle set $\cycleSet$ and
will be called the zero cyles set and the one cycles set respectively.

\addExample{cycle_partition.tex}

\section{Solving Individual Permutation Cycles}
\label{sec:solvindpermcycles}

This division of the cycle set $\cycleSet$ is important to correctly identify which
of the methods to be presented need to be used.
Any cycle $\cycle$ of  $\cycleSet$ have is an element of a partition of the vertices $\partitionFunc
{\cycle}$ created by the subtrees of the lowest common ancestor node of the cycle, 
as seen in Figure~\ref{img:cycle_partition}. 
Per the definition of the cotree, a partition of a cycle of $\zeroCycles$ have no 
edges between any of these partitions in $\graphG$.
Similarly, any cycle of the one cycles set $\oneCycles$ have a partition of the
vertices such that all edges exists between each element of the partition in $\graphG$.
Each partition element represents the leafs in each of the subtrees of the lowest 
common ancestor node and the union of all partitions of $\partitionFunc{\cycle}$ is 
exactly $\vertexFunc{\cycle}$.
These are the fundamental properties used in Lemma~\ref{lem:s_onecycle} and 
Lemma~\ref{lem:s_zerocycle}, in which the proofs presents a method for solving 
each cycle type with a determined number of swaps.
An individual permutation cycle $\cycle$ is a instance of the TS where every other
permutation cycle is a cycle of size one in the same graph as the original
instance.

% Botar exemplo(s) de resolvendo ciclos parte of cO e c1 aqui.

\begin{lemma}
\label{lem:s_onecycle}
Any cycle $\cycle$ of $\oneCycles$ can be solved in $\sizeof{\cycle} - 1$ swaps.
\end{lemma}

\begin{proof}
Let $\partitionFunc{\cycle} = \createSet{\partitionElem_{1}, \partitionElem_{2}, 
\ldots, \partitionElem_{k}}$ be the partition created by the lowest common
ancestor node.
Induction on the size of the cycle will be used to prove this lemma.
Assume that all other cycles of size $m$, with $m < \sizeof{\vertexFunc{\cycle}}$ 
can be solved with $m - 1$ swaps.
The base is a cycle of size one that is already solved, as $1-1 = 0$.
For the induction step, a argument on the size of the partitions will be used.
There are two cases:

Case 1: Let $\partitionElem = \createSet{\vertexv}$ be a element of size 
one of $\partitionFunc{\cycle}$.
Then, $\vertexv$ dominates the vertices of the $\vertexFunc{\cycle}$ that he 
is part of, as all edges between partitions \textit{must} exist, by the cograph
definition.
Swap the token on $\vertexv$ to his target vertice, splitting this cycle in
one of size one and other of size $m - 1$.
The cycle of size one is already solved, while the cycle of size $m - 1$ can
be solved in $m - 2$ by the induction hypothesis, resulting in $m - 1$ swaps
to solve the original cycle;
% Note that the resulting cycle will still have the partition of size one, as
% vertice $\vertexv$ will be the last to be solved in the cycle;

Case 2: If there is no element of size one, it is possible to create an element
of size one while maintaining the desired number of swaps by iteratively 
solving separated nodes and removing them from a partition to achieve a 
partition of size one and go to case number 1.
Choose any element of the partition $\partitionElem \in \partitionFunc{\cycle}$.
By the definition of the cycle, there must exist at least one edge of the cycle
that goes from $\partitionElem$ to some other element of $\partitionFunc{\cycle}$
and at least one edge that goes to $\partitionElem$ from some other element of
$\partitionFunc{\cycle}$.
Let $\createOrd{\vertexu, \vertexw}$ be any of these directed edges of the permutation 
cycle that goes \textbf{to} $\partitionElem$.
A swap applied along this edge will create two resulting permutation cycles: One 
of size one and one of size $\sizeof{\cycle} - 1$, effectively moving one token 
to the correct vertex and removing this node from the original cycle $\cycle$.
Let this new cycle be called $\sub{\cycle}$.
The partition $\partitionFunc{\sub{\cycle}}$ is exactly the partition of the 
original $\cycle$, with the element $\partitionElem$ with one vertice less,
the vertice $\vertexw$.
By the induction hypothesis, these two resulting cycles can be solved with
the correct number of swaps, resulting in $\sizeof{\cycle} - 1$ swaps.
Note that the resulting cycle of size $\sizeof{\cycle} - 1$ will fall into either
Case 2 or Case 1 eventually.
\end{proof}

% \begin{proof}
% Let $\partitionFunc{\cycle} = \createSet{\partitionElem_{1},\partitionElem_{2},
% \ldots,\partitionElem_{k}}$ be the partition created by the lowest common ancestor
% node.
% There are three cases that needs to be analyzed: When $\sizeof{\partitionFunc{\cycle}} 
% = 1$; When $\sizeof{\partitionFunc{\cycle}} > 1$ but there is at least one partition 
% $\partitionElem \in \partitionFunc{\cycle}$ such that $\sizeof{\partitionElem} = 1$;
% And when $\sizeof{\partitionFunc{\cycle}} > 1$ and there is no partition 
% $\partitionElem \in \partitionFunc{\cycle}$ of size 1.

% Case 1: There is no need to add a swap, as this is a cycle of size one, already correct.
% So, $\sizeof{\cycle} - 1 = 1 - 1 = 0$.

% Case 2: Let $\partitionElem = \createSet{\vertexv}$ be the partition of size 
% one of $\partitionFunc{\cycle}$.
% Then, $\vertexv$ dominates the vertices of the $\vertexFunc{\cycle}$ that he 
% is part of, as all edges between partitions \textit{must} exist, by the cograph
% definition.
% Use the vertex $\vertexv$ as a center of a start and separate the star subgraph
% that is embedded in his neighborhood.
% Create a separated instance of the token swap problem where the graph is a 
% star and there is exactly one permutation cycle that involves the center 
% $\vertexv$.
% Remember by Section~\ref{sec:intro:classes_uppers:star} that the minimum number
% of swaps of a star is given by $n + l$, where $n$ is the number of unhappy 
% leaves and $l$ is the number of cycles in the permutation with length at least 
% 2 that does not involve the center node.
% There is exactly $\sizeof{\cycle} - 1$ unhappy leaves and no more cycles that
% does not involves the center node, resulting in exactly $\sizeof{\cycle} - 1$
% swaps by using the star method in this created instance.
% This same swaps can be used in the original token swap instance, as they are
% independent.

% Case 3: If there is no partition of size one, it is possible to create a partition
% of size one while maintaining the desired number of swaps by iteratively 
% solving separated nodes and removing them from a partition to achieve a 
% partition of size one and go to case number 2.
% Choose any partition $\partitionElem \in \partitionFunc{\cycle}$.
% By the definition of the cycle, there must exist at least one edge of the cycle
% that goes from $\partitionElem$ to some other partition of $\partitionFunc{\cycle}$
% and at least one edge that goes to $\partitionElem$ from some other partition of
% $\partitionFunc{\cycle}$.
% Let $\createOrd{\vertexu, \vertexw}$ be this directed edge of the permutation cycle
% that goes \textbf{to} $\partitionElem$.
% By the split rule, a swap applied in along this edge will create two resulting
% permutation cycles: One of size one and one of size $\sizeof{\cycle} - 1$, 
% effectively moving one token to the correct vertice and removing this node from
% the original cycle $\cycle$.
% Let this new cycle be called $\sub{\cycle}$.
% The partition $\partitionFunc{\sub{\cycle}}$ is exactly the partition of the 
% original $\cycle$, with the partition $\partitionElem$ with one vertice less.
% Use this process to iteratively remove a node until $\sizeof{\partitionElem} = 1$
% and then apply the Case 2 to the resulting cycle.
% Let $d$ be the number of swaps used to "shrink" the cycle $\cycle$ to a cycle 
% $\sub{\cycle}$ that has a partition of size one.
% As each swap solved exactly one vertice, the size of the new cycle is 
% $\sizeof{\sub{\cycle}} = \sizeof{\cycle} - d$ and can be solved with $\sizeof{\cycle} 
% - d - 1$ swaps.
% The sum of all swaps used to solve $\cycle$ is then $\sizeof{\cycle} - d - 1 + d = 
% \sizeof{\cycle} - 1$.
% Note that the lowest common ancestor does not change, as not enough vertices
% are removed from the partitions to remove an entire subtree of the cograph.
% \end{proof}

% Add one more figure here

\begin{lemma}
\label{lem:s_zerocycle}
Any cycle $\cycle$ of $\zeroCycles$ can be solved in $\sizeof{\cycle} + 1$ swaps.
\end{lemma}

\begin{proof}
Let $\partitionFunc{\cycle} = \createSet{\partitionElem_{1}, \partitionElem_{2}, 
\ldots, \partitionElem_{k}}$ be the partition created by the lowest common
ancestor node.
By the cograph definition, each element of $\partitionFunc{\cycle}$ have no
edges between them, resulting in no possible path to move tokens from one 
element to another by only using the nodes $\vertexFunc{\cycle}$.
In order to solve $\cycle$, at least one node must be \textit{merged} to cycle 
$\cycle$ so it can be solved as a cycle of $\oneCycles$.

As $\graphG$ is connected, there must exist at least one ancestor 1-node that 
connects the graph (or more locally, the nodes of $\vertexFunc{\graphG}$).
Let the ancestor node called $\vertexw$ and let
$\nodePartition{\vertexw}$ be the partition of the leafs of the children 
subtrees of $\vertexw$ in the cotree $\cotreeOf{\graphG}$.
There must exist a $\partitionElem \in \nodePartition{\vertexw}$ such that 
$\vertexFunc{\cycle} \subseteq \partitionElem$, as $\vertexw$ must be 
a direct ancestor of $\lcaOf{\cotreeOf{\graphG}}{\cycle}$.
Let $\vertexu$ be any node that is an element of any set in $\partitionFunc{\cycle} 
\setminus \partitionElem$.
Then, $\vertexu$ must dominate $\vertexFunc{\cycle}$.

Now, \textit{assume} that $\vertexu$ already has a correct token in the current 
mapping and merge him to the cycle $\cycle$ by using any swap $\createOrd{\vertexu,
\vertexv}$, $\vertexv \in \vertexFunc{\cycle}$, and call this new cycle $\sub{\cycle}$.
In this new cycle, the 1-node $\vertexw$ is the lowest common ancestor and the
technique presented at Lemma~\ref{lem:s_onecycle} can be used to solve in 
$\sizeof{\sub{\cycle}} - 1$ swaps, which is equivalent to $\sizeof{\cycle}$
swaps, as there is one vertice more. 
In total, the number of swaps is $\sizeof{\cycle} + 1$ because of the first swap 
needed to add $\vertexu$ to the cycle.
Note that as the token of $\vertexu$ is assumed to be correct, his token went
back to the same place as before, not changing any other cycle of the
configuration.
This means that the token currently in $\vertexu$ does not need to be the correct
token.
\end{proof}

With both of the methods presented, all permutation cycles of any TS instance
on a cograph $\graphG$ can be solved with a value that can be easily calculated 
and is denoted as $\sizeof{\vertexFunc{\graphG}} - \sizeof{\oneCycles} + 
\sizeof{\zeroCycles}$.
Now, some exploration can be done about the strength of this characterization
and how small are these values of swaps for these cycles.
First, it is possible to show that the methods can achieve the minimum
number of swaps for individual cycles by Lemma~\ref{lem:m_onecycle} and
Lemma~\ref{lem:m_zerocycle}.

\begin{lemma}
\label{lem:m_onecycle}
Individually, any cycle $\cycle$ of $\oneCycles$ cannot be solved in less 
than $\sizeof{\cycle} - 1$ swaps.
\end{lemma}

\begin{proof}
Suppose that the cycle $\cycle$ can be solved in less than $\sizeof{\cycle} 
- 1$ swaps.
Then, suppose an instance of TS where the graph is a complete graph with
the same number of vertices and labels of the original cograph.
Let the permutation cycle $\cycle$ be the only cycle of size greater than 
1 in the initial configuration of this instance.
The optimal number of swaps in this instance is described in Section~\ref
{sec:intro:classes_uppers:clique} and is given by $n - r$, where $n$ is the
number of vertices and $r$ is the total number of permutation cycles.
In our instance, it is possible to see that the minimum number of swaps
is $n - 1$ even with a complete topology, showing that the same cycle
on a more restricted topology cannot be possibly less than $n - 1$, as
the original graph is a subgraph of the complete topology.
\end{proof}

\begin{lemma}
\label{lem:lb_merge_swaps}
Let $\graphG$ be a graph, $\initMap$ an initial configuration and $\cycle$ any
permutation cycle on this instance.
Let $k$ be the minimum number of merge swaps needed to solve the individual 
instance of $\cycle$ in any swap sequence $\swapSeq$.
Then, $\sizeof{\swapSeq} \geq \sizeof{\cycle} - 1 + 2k$.
\end{lemma}

\begin{proof}
The reasoning is based on the number of permutation cycles on the instance.
Let $s$ be the minimum number of split swaps necessary to solve the instance
and $\sizeof{\vertexFuncG} - \sizeof{\cycle} + 1$ the total number of cycles
in this instance.
The target identity mapping must have $\sizeof{\vertexFuncG}$ cycles to be
correct, so, it needes at least $\sizeof{\cycle} - 1$ split swaps, as
every split swap must increase the number of cycles by one.
However, it is known that at least $k$ merge swaps are needed to solve this
instance, which increases the total number of cycles by $k$ at total.
Therefore, the number of split swaps must be $s \geq \sizeof{\cycle} - 1 + k$.
Thus, the total number of swaps is bounded by $\sizeof{\swapSeq} = s + k \geq 
\sizeof{\cycle} - 1 + 2k$.
\end{proof}

\begin{lemma}
\label{lem:m_zerocycle}
Individually, any cycle $\cycle \in \zeroCycles$ cannot be solved in less
than $\sizeof{\cycle} + 1$ swaps.
\end{lemma}

%trash proof, revise this

\begin{proof}
% Suppose that the cycle $\cycle$ can be solved in less than $\sizeof{\cycle}
% + 1$ swaps.
% As the partitions of $\partitionFunc{\cycle}$ are disjoint, there is no
% path to move tokens between partitions by only using vertices of 
% $\vertexFunc{\cycle}$, so at least one token must be used as a auxiliary
% to solve this cycle.
% By Lemma~\ref{lem:s_zerocycle}, just one more vertice is enough to solve 
% any cycle in $\zeroCycles$, which adds exactly two swaps to the final
% swap sequence.
% From Lemma~\ref{lem:m_onecycle}, the number of swaps in the individual 
% instance of this cycle with one more vertice is already optimal.
% Exactly one swap is used to merge a new vertice into the cycle, meaning
% that there is no place where one swap could be saved.
By Lemma~\ref{lem:s_zerocycle}, every $\cycle \in \zeroCycles$ can be solved
by using exactly one merge swap, as every other swap given by the method
on Lemma~\ref{lem:s_onecycle} are split swaps (swaps along an edge of the
permutation cycle).
Thus, by Lemma~\ref{lem:lb_merge_swaps}, as the minimum number of merge 
swaps $k$ needed for cycles of $\zeroCycles$ is one, each cycle 
is lower bounded by $\sizeof{\cycle} - 1 + 2\times1 = \sizeof{\cycle} + 1$, 
which is exactly the amount of swaps given by the method shown. 
\end{proof}

This model of solving cycles can be used to prove the optimality of some
subclasses of the cograph class.
In complete graphs, every permutation cycle belongs to $\oneCycles$, as
every vertice is connected to every other vertice, resulting in an empty
$\zeroCycles$ set.
In star graphs, the number of cycles $\cycle \in \oneCycles$ with 
$\sizeof{\cycle} > 1$ cannot be more than 1, as all nodes are only 
neighbors of the center node and the center node can be only part 
of one cycle, resulting in every other non-trivial cycle belonging to 
$\zeroCycles$.
These two classes can be generalized to the class of threshold graphs, 
as seen in Lemma~\ref{cor:threshold_pol}.
There are many equivalent definitions for this class, but, for the sake
of conciseness, a threshold graph is a graph that can be constructed from
the empty graph by repeatedly adding either an isolated vertice or a
dominating vertice.
Also, a Threshold graph is a graph free of induced cycles of size four 
($C_{4}$), induced paths of size four ($P_{4}$) and induced two disjoint 
edges ($2K_{2}$).

% Put an image of the related cotrees

\begin{lemma}
\label{lem:ub_threshold}
Let $\graphG$ be a threshold graph with an initial token placement $\initMap$.
Then, $\applyFunc{OPT_{\graphG}}{\initMap} \leq \sizeof{\vertexFunc{\graphG}} - 
\sizeof{\applyFunc{\oneCycles}{\conflictGraph_{\mapFunc}}} +
\sizeof{\applyFunc{\zeroCycles}{\conflictGraph_{\mapFunc}}}$.
\end{lemma}

\begin{proof}
Directly from Lemma~\ref{lem:m_onecycle} and Lemma~\ref{lem:m_zerocycle}.
\end{proof}

\begin{lemma}
\label{lem:lb_threshold}
Let $\graphG$ be a threshold graph with an initial token placement $\initMap$.
Then, $\applyFunc{OPT_{\graphG}}{\initMap} \geq \sizeof{\vertexFunc{\graphG}} - 
\sizeof{\applyFunc{\oneCycles}{\conflictGraph_{\mapFunc}}} +
\sizeof{\applyFunc{\zeroCycles}{\conflictGraph_{\mapFunc}}}$.
\end{lemma}

\begin{proof}
Let $\applyFunc{p}{\graphG, \mapFunc} = \sizeof{\vertexFunc{\graphG}} - 
\sizeof{\applyFunc{\oneCycles}{\conflictGraph_{\mapFunc}}} +
\sizeof{\applyFunc{\zeroCycles}{\conflictGraph_{\mapFunc}}}$.
Note that $\applyFunc{p}{\graphG, \idMap} = 0$, the number of swaps needed to
solve the identity configuration, holds.
First, it is going to be necessary to prove that for any swap on $\mapFunc$,
the value of $\applyFunc{p}{\graphG, \sub{\mapFunc}}$ will be decreased
by at most one, such that $\sub{\mapFunc}$ is a adjacent configuration
of $\mapFunc$.
The following cases encompass every possible split interaction between 
cycles.

\begin{itemize}
    \item \textbf{Case SPLIT-1:} Let $\vertexu$ and $\vertexv$ belong to 
    the same cycle $\cycle \in \oneCycles$. 
    This token swap breaks the original cycle into two vertex disjoint cycles that
    we will call $\cycle_{\vertexu}$ and $\cycle_{\vertexv}$.
    Assume that $\vertexu \in \vertexFunc{\cycle_{\vertexu}}$ and 
    $\vertexv \in \vertexFunc{\cycle_{\vertexv}}$.
    
    \begin{enumerate}
        \item[] \textbf{Case (A):} If $\cycle_{\vertexu} \in \oneCycles$ and 
        $\cycle_{\vertexv} \in \oneCycles$, the value of $\sizeof{\oneCycles}$ 
        is increased in 1, resulting in $\applyFunc{p}{\graphG, \sub{\mapFunc}} =
        \applyFunc{p}{\graphG, \mapFunc} - 1$;
        \item[] \textbf{Case (B):} If  $\cycle_{\vertexu} \in \zeroCycles$ and 
        $\cycle_{\vertexv} \in \oneCycles$, the value of $\sizeof{\zeroCycles}$ 
        is increased in 1, with $\sizeof{\oneCycles}$ remaining unchanged, 
        resulting in $\applyFunc{p}{\graphG, \sub{\mapFunc}} = \applyFunc{p}
        {\graphG, \mapFunc } + 1$;
        \item[] \textbf{Case (C):} Assume $\cycle_{\vertexu},\cycle_{\vertexv} 
        \in \zeroCycles$.
        There must exist two elements of the partition $\partitionElem_{\vertexu},
        \partitionElem_{\vertexv}$ of $\partitionFunc{\cycle}$ such that
        $\vertexFunc{\cycle_{\vertexu}} \subseteq \partitionElem_{\vertexu}$
        and $\vertexFunc{\cycle_{\vertexv}} \subseteq \partitionElem_{\vertexv}$.
        Take any two disconnected vertices $x,y$ from $\vertexFunc{\cycle_{\vertexu}}$
        and two disconnected vertices $w,z$ from $\vertexFunc{\cycle_{\vertexv}}$.
        Note that they must exist on each cycle as they are a zero cycle. 
        Then, the induced subgraph $\inducedSubgraph{\graphG}{\createSet{x,y,w,z}}$ on
        the original graph is a cycle of size 4.
        This is a contradiction on the definition of the Threshold graph. 
    \end{enumerate}
    
    \item \textbf{Case SPLIT-0:} Let $\vertexu$ and $\vertexv$ belong to 
    the same cycle $\cycle \in \zeroCycles$. 
    This token swap breaks the original cycle into two vertex disjoint cycles that
    we will call $\cycle_{\vertexu}$ and $\cycle_{\vertexv}$.
    Assume that $\vertexu \in \vertexFunc{\cycle_{\vertexu}}$ and 
    $\vertexv \in \vertexFunc{\cycle_{\vertexv}}$.
    
    \begin{enumerate}
        \item[] \textbf{Case (A):} Let $\cycle_{\vertexu},\cycle_{\vertexv} \in \oneCycles$.
        Then, the total number of swaps given by our model is $\sizeof{\cycle_{\vertexu}} - 1 + 
        \sizeof{\cycle_{\vertexv}} - 1 + 1 = \sizeof{\cycle_{\vertexu}} + 
        \sizeof{\cycle_{\vertexv}} - 1 = \sizeof{\cycle} - 1$.
        This result is a contradiction on Lemma~\ref{lem:m_zerocycle}, as this cycle
        cannot be solved individually in less than $\sizeof{\cycle} + 1$ swaps;
        \item[] \textbf{Case (B):} If $\cycle_{\vertexu} \in \zeroCycles$ and 
        $\cycle_{\vertexv} \in \oneCycles$, the value of $\sizeof{\oneCycles}$ 
        is increased by 1 and $\sizeof{\zeroCycles}$ continues the same, 
        resulting in $\applyFunc{p}{\graphG, \sub{\mapFunc}} = \applyFunc{p}
        {\graphG, \mapFunc} - 1$;
        \item[] \textbf{Case (C):} If $\cycle_{\vertexu} \in \zeroCycles$ and 
        $\cycle_{\vertexv} \in \zeroCycles$, the value of $\sizeof{\zeroCycles}$ 
        is increased by 1, resulting in $\applyFunc{p}{\graphG, \sub{\mapFunc}} 
        = \applyFunc{p}{\graphG, \mapFunc} + 1$;
    \end{enumerate}
    
    \end{itemize}
    
    It is not needed to check the merge swaps in this case, as they are just the inverse
    of a split.
    This means that the equations have inverted plus and minus signals, and, as no equation
    has other value greater than one, the proof that $\applyFunc{p}{\graphG, \mapFunc}$
    decreases by at maximum one for every possible swap holds.
    By the above analysis, it is possible to conclude that any token swap decreases
    $\applyFunc{p}{\graphG, \mapFunc}$ by at most one for any token placement $\mapFunc$ and
    obtain the Inequation~\ref{equ:decrease_by_one}.
    
    \begin{equation}
        \centering
        \label{equ:decrease_by_one}
        \applyFunc{p}{\graphG, \sub{\mapFunc}} \geq \applyFunc{p}{\graphG, \mapFunc} - 1
    \end{equation}
    
    Thus, for any swapping sequence $\swapSeq = \createOrd{\swaps_1, \swaps_2, \ldots, 
    \swaps_k}$ that transforms the initial configuration $\initMap$ to the identity
    configuration $\idMap$ through adjacent configurations $\mapFunc_{1}, \mapFunc_2,
    \ldots, \mapFunc_{k} = \idMap$, each pair of configurations $\applyFunc{p}{\graphG, 
    \mapFunc_{j+1}} \geq \applyFunc{p}{\graphG, \mapFunc_{j}} -1$ holds from 
    Inequation~\ref{equ:decrease_by_one} for $j = 1,2,\ldots,k-1$.
    Take the sum of these inequations $\sum_{j} \applyFunc{p}{\graphG, \mapFunc_{j+1}} \geq \applyFunc{p}{\graphG, \mapFunc_{j}} -1$ shown in Inequation~\ref{equ:equation_sum}.
    
    \begin{alignat}{2}
    \centering
    \label{equ:equation_sum}
    & \applyFunc{p}{\graphG, \mapFunc_{1}}  \quad && \geq \quad \applyFunc{p}{\graphG, \mapFunc_{0}} -1 \nonumber \\
    & \applyFunc{p}{\graphG, \mapFunc_{2}} \quad && \geq \quad \applyFunc{p}{\graphG, \mapFunc_{1}} -1 \nonumber \\
    & \ldots \nonumber \\
    & \applyFunc{p}{\graphG, \mapFunc_{k-1}} \quad && \geq \quad \applyFunc{p}{\graphG, \mapFunc_{k}} -1 \nonumber \\
    \cline{1-4}
    & \applyFunc{p}{\graphG, \mapFunc_{k}} \quad && \geq \quad \applyFunc{p}{\graphG, \mapFunc_{0}} - \sizeof{\swapSeq}
    \end{alignat}
    
    By organizing Inequation~\ref{equ:equation_sum}, substituting $\applyFunc{p}{\graphG, 
    \mapFunc_{k}}$ for a $0$, as it is the last configuration and this swap sequence solve the
    instance, and substituting $\applyFunc{p}{\graphG, \mapFunc_{0}}$ by the original assumption,
    the Inequation~\ref{equ:lower_bound} is found.
    
    \begin{alignat}{2}
    \centering
    \label{equ:lower_bound}
    & \applyFunc{p}{\graphG, \mapFunc_{k}} \quad && \geq \quad \applyFunc{p}{\graphG, \mapFunc_{0}} - \sizeof{\swapSeq} \nonumber \\
    & \sizeof{\swapSeq} \quad && \geq \quad \applyFunc{p}{\graphG, \mapFunc_{0}} - \applyFunc{p}{\graphG, \mapFunc_{k}} \nonumber \\
    & \sizeof{\swapSeq} \quad && \geq \quad \sizeof{\vertexFunc{\graphG}} - 
\sizeof{\applyFunc{\oneCycles}{\conflictGraph_{\mapFunc}}} +
\sizeof{\applyFunc{\zeroCycles}{\conflictGraph_{\mapFunc}}}
    \end{alignat}
    
\end{proof}

\begin{theorem}
\label{cor:threshold_pol}
Let $\graphG$ be a threshold graph with an initial token placement $\initMap$.
The minimum number of required swaps is given by $\sizeof{\vertexFunc
{\graphG}} - \sizeof{\oneCycles} + \sizeof{\zeroCycles}$.
\end{theorem}

\begin{proof}
Directly from Lemma~\ref{lem:ub_threshold} and Lemma~\ref{lem:lb_threshold}.
\end{proof}

\section{Dependencies Betweeen Permutation Cycles}
\label{sec:deppermcycles}

Section~\ref{sec:solvindpermcycles} gave techniques for solving the two types of 
cycles individually without changing the configurations of other cycles and proved 
the optimality of this method for Threshold graphs.
The proof for the Cographs is constructed from the Treshold graphs, based on the 
observation that Cographs allow cycles of size 4.
When true, a merge of two zero-cycles to a one-cycle can save exactly two swaps
in the final configuration.
Let this swap be called a \textit{cutback} swap.

\begin{lemma}
\label{lem:zero_split}
Let $\graphG$ be a cograph with an  token placement $\mapFunc$.
Let $\cycle_{1}$ and $\cycle_{2}$ be the two cycles resulting from any split swap 
on a cycle $\cycle$ member of $\zeroCycles$. 
Either $\cycle_{1}$ or $\cycle_{2}$ must be a cycle with exactly the same lowest 
common ancestor as the original cycle $\cycle$.
\end{lemma}

\begin{proof}
Let $\vertexSet_{1}, \ldots, \vertexSet_{k}$ be elements of a partition 
$\partitionFunc{\vertexFunc{\cycle}}$ where $k$ is the number of children 
of the node $\lcaOf{\applyFunc{\conflictGraph}{\graphG}}{\vertexFunc{\cycle}}$
in the cotree of $\graphG$ and each element corresponds to the the set of
all graph $\graphG$ nodes present in the children subtree.
Assume that neither $\cycle_{1}$ or $\cycle_{2}$ have the same lowest common
ancestor as $\cycle$.
Note that by the definition of a 0-cycle this swap must happen between two
nodes that are member of the same element of the partition $\partitionFunc
{\vertexFunc{\cycle}}$, as there is no edges between each element of the 
partition, $\vertexFunc{\cycle_{1}} \cup \vertexFunc{\cycle_{2}} = 
\vertexFunc{\cycle}$ and each partition element of must have \textit{at least 
one} token positioned at a vertice that needs to be moved to another partition 
element, as these vertices forms a cycle.

If $k > 2$, then by the pigeon-hole principle~\citep{Erdos:1987,Razborov:2002}, 
either $\cycle_{1}$ or $\cycle_{2}$ (or both) have vertices from more than 
one element of the partition, resulting in the same lowest common ancestor 
as $\cycle$.
If $k = 2$, then $\cycle_{1},\cycle_{2}$ must be each an element of 
$\partitionFunc{\vertexFunc{\cycle}}$.
Therefore, there must exist at least one swap between the two elements that
perfectly split the two elements to two disjoint cycles.
But, as this is a 0-cycle, no such swap can exist.
If $k = 1$, then there is a contradiction on the common ancestor, because
there is a lower common ancestor to be found.
\end{proof}

A \textit{Cycle Matching Graph} $\cycleMatchingGraph = \createOrd{\zeroCycles, 
\edgeSet_{\cycleMatchingGraph}}$ of a cograph $\graphG$ and initial token 
configuration $\initMap$ is a graph where each node represents an individual zero
cycle of the current configuration. 
Let $\vertexu, \vertexv \in \vertexFunc{\cycleMatchingGraph}$ and $\cycle_{\vertexu}, 
\cycle_{\vertexv}$ be the two related cycles from $\zeroCycles$.
An edge will exist between two nodes $\vertexu,\vertexv$ if and only if the lowest 
common ancestor of the union of the vertices $\vertexFunc{\cycle_{\vertexu}} \cup
\vertexFunc{\cycle_{\vertexv}}$ is a one-node.
This graph will be used to identify these cases where two swaps can be saved.

\begin{lemma}
\label{lem:matching_inc_one}
Let $\graphG$ be a graph and $\matchingOf{\graphG}$ a maximum edge matching.
\begin{enumerate}
    \item Adding a vertice connected by edges to the graph $\graphG$ can increase the size of the matching 
$\sizeof{\matchingOf{\graphG}}$ in at most one; %\label{lem:matching_inc_one:1}
    \item Adding non-existing undirected edges to a vertice can increase the size of the matching 
$\sizeof{\matchingOf{\graphG}}$ in at most one; %\label{lem:matching_inc_one:2}
    \item Let $\sub{\graphG}$ and $\ssub{\graphG}$ be graphs such that $\graphG \subseteq \sub{\graphG}
    \subseteq \ssub{\graphG}$, $\sizeof{\vertexFunc{\sub{\graphG}} \setminus \vertexFunc{\graphG}} = 1$
    and $\sizeof{\vertexFunc{\ssub{\graphG}} \setminus \vertexFunc{\sub{\graphG}} } = 1$.
    Then, $\sizeof{\matchingOf{\ssub{\graphG}}} \leq \sizeof{\matchingOf{\graphG}} + 2$.
\end{enumerate}
\end{lemma}

\begin{proof}
The proofs are respectively enumerated below.
    \begin{enumerate}
        \item Let $\sub{\graphG}$ be the graph resulted by adding a vertice $x$ with any 
adjacency and let $\matchingOf{\sub{\graphG}}$ be its maximum edge matching.
Suppose $x$ is saturated in $\matchingOf{\sub{\graphG}}$ and $\sizeof{\matchingOf
{\sub{\graphG}}} > \sizeof{\matchingOf{\graphG}}$. 
In special, $\sizeof{\matchingOf{\sub{\graphG}}}$ must be $\sizeof{\matchingOf
{\graphG}} + 1$, otherwise the size of the maximum matching $\sizeof{\matchingOf
{\graphG}}$ results in a contradiction, as it is possible to use the matching 
$\sizeof{\matchingOf{\sub{\graphG}}}$ minus the edge that $x$ participates as a 
new bigger matching.
If $x$ is not saturated, then $\sizeof{\matchingOf{\sub{\graphG}}} = \sizeof
{\matchingOf{\graphG}}$ by the same reason as above;
        \item Let $\sub{\graphG}$ be the graph resulted by adding any amount of non-existing 
edges to a vertice $\vertexv \in \vertexFuncG$ and let $\matchingOf{\sub{\graphG}}$ be it's 
maximum edge matching.
Let $\matchingOf{\sub{\graphG}}$ be matching of size greater than $\matchingOf{\graphG} + 1$.
the resulting matching created by removing the edge matching related to $\vertexv$ in 
$\matchingOf{\sub{\graphG}}$ is a valid matching on the original graph $\graphG$ and has
size at least $\matchingOf{\graphG} + 1$, creating a contradiction on the size of the maximum
matching of $\graphG$.
        \item By Lemma~\ref{lem:matching_inc_one}, the size of the matching 
        $\matchingOf{\sub{\graphG}}$ can increase in at most 1 from $\matchingOf{\graphG}$,
        as in Inequation~\ref{equ:matching_sum:1}, and the size of the matching $\matchingOf
        {\ssub{\graphG}}$ can increase in at most 1 from $\matchingOf{\sub{\graphG}}$, as in
        Inequation~\ref{equ:matching_sum:2}.
        By summing these two inequations, the Inequation~\ref{equ:matching_sum:3} is achieved.
        
            \begin{alignat}{2}
            \centering
            &\matchingOf{\sub {\graphG}} \leq \matchingOf{\graphG} + 1 \label{equ:matching_sum:1}\\
            &\matchingOf{\ssub{\graphG}} \leq \matchingOf{\sub{\graphG}} + 1 \label{equ:matching_sum:2}\\ 
            \cline{1-2}
            &\matchingOf{\ssub{\graphG}} \leq \matchingOf{\graphG} + 2 \label{equ:matching_sum:3}
            \end{alignat}
    \end{enumerate}
\end{proof}

Let $\matchingOf{\cycleMatchingGraph}$ be a maximum edge matching on the graph 
$\cycleMatchingGraph$.
Each element of this matching represents a possible merge swap that can save two 
swaps in the final swap sequence, saving in total $2\times\sizeof{\matchingOf
{\cycleMatchingGraph}}$ swaps.
Note that after applying these swaps, the resulting configuration will have a
totally disconnected Cycle Matching Graph, as only cycles unsaturated by the 
matching $\matchingOf{\cycleMatchingGraph}$ will stay.

% \begin{lemma}
% Let $\graphG$ be a cograph with an initial token placement $\initMap$ and cycle matching 
% graph $\cycleMatchingGraph$.
% It is possible to save at maximum $2\times\sizeof{\matchingOf
% {\cycleMatchingGraph}}$ swaps by using cutback swaps.
% \end{lemma}

% \begin{proof}
% Assume a way to save more than $2\times\sizeof{\matchingOf{\cycleMatchingGraph}}$ swaps.
% First, consider the initial token placement $\initMap$ and each possible cutback swap is 
% identified by a cycle matching graph $\cycleMatchingGraph$.
% If there is more than $\sizeof{\matchingOf{\cycleMatchingGraph}}$ cutback swaps on
% $\initMap$ and they do not share a vertice, then there is a contradiction on size of
% the maximum matching $\matchingOf{\cycleMatchingGraph}$.
% Otherwise, if they share a vertice, the new set of cutback swaps are invalid as for each 
% swap the related vertices on $\cycleMatchingGraph$ are removed when applied, making the swap
% not a cutback swap.

% Now, there is an achievable configuration $\mapFunc$ with cycle matching graph 
% $\sub{\cycleMatchingGraph}$ to our current configuration such that $\sizeof{\matchingOf
% {\sub{\cycleMatchingGraph}}} - A > \sizeof{\matchingOf{\cycleMatchingGraph}}$? The value
% $A$ is the number of additional swap necessary to move to configuration $\mapFunc$.
% \end{proof}

% \begin{lemma}
% Let $\graphG$ be a cograph with an initial token placement $\initMap$ and cycle matching 
% graph $\cycleMatchingGraph$.
% Any optimal swap sequence $\swapSeq$ that solves this instance will have at maximum
% $\sizeof{\matchingOf{\cycleMatchingGraph}}$ cutback swaps.
% \end{lemma}

% \begin{proof}
% Let $\applyFunc{p}{\graphG, \mapFunc} = \sizeof{\matchingOf{\cycleMatchingGraph}}$.
% Note that $\applyFunc{p}{\graphG, \idMap} = 0$ holds.
% First, it is going to be necessary to prove that for any swap on $\mapFunc$,
% the value of $\applyFunc{p}{\graphG, \sub{\mapFunc}}$ will be decreased
% by at maximum one, such that $\sub{\mapFunc}$ is a adjacent configuration
% of $\mapFunc$.
% The following cases encompass every possible split interaction between 
% cycles.

% \begin{itemize}
%     \item \textbf{Case SPLIT-1:} Let $\vertexu$ and $\vertexv$ belong to 
%     the same cycle $\cycle \in \oneCycles$. 
%     This token swap breaks the original cycle into two vertex disjoint cycles that
%     we will call $\cycle_{\vertexu}$ and $\cycle_{\vertexv}$.
%     Assume that $\vertexu \in \vertexFunc{\cycle_{\vertexu}}$ and 
%     $\vertexv \in \vertexFunc{\cycle_{\vertexv}}$.
    
%     \begin{enumerate}
%         \item[] \textbf{Case (A):} If $\cycle_{\vertexu} \in \oneCycles$ and 
%         $\cycle_{\vertexv} \in \oneCycles$, the graph $\cycleMatchingGraph$
%         cannot change as no cycle of $\oneCycles$ has been modified, resulting
%         in $\applyFunc{p}{\graphG, \sub{\mapFunc}} = \applyFunc{p}{\graphG, \mapFunc}$;
%         \item[] \textbf{Case (B):} If  $\cycle_{\vertexu} \in \zeroCycles$ and 
%         $\cycle_{\vertexv} \in \oneCycles$, then exactly one vertice can be added
%         to $\vertexFunc{\cycleMatchingGraph}$, which, by Lemma~\ref{lem:matching_inc_one},
%         can add at maximum one edge to the matching;
%         \item[] \textbf{Case (C):} Assume $\cycle_{\vertexu},\cycle_{\vertexv} 
%         \in \zeroCycles$. 
%     \end{enumerate}
    
%     \item \textbf{Case SPLIT-0:} Let $\vertexu$ and $\vertexv$ belong to 
%     the same cycle $\cycle \in \zeroCycles$. 
%     This token swap breaks the original cycle into two vertex disjoint cycles that
%     we will call $\cycle_{\vertexu}$ and $\cycle_{\vertexv}$.
%     Assume that $\vertexu \in \vertexFunc{\cycle_{\vertexu}}$ and 
%     $\vertexv \in \vertexFunc{\cycle_{\vertexv}}$.
    
%     \begin{enumerate}
%         \item[] \textbf{Case (A):} Let $\cycle_{\vertexu},\cycle_{\vertexv} \in \oneCycles$.
%         Then, the total number of swaps is $\sizeof{\cycle_{\vertexu}} - 1 + 
%         \sizeof{\cycle_{\vertexv}} - 1 + 1 = \sizeof{\cycle_{\vertexu}} + 
%         \sizeof{\cycle_{\vertexv}} - 1 = \sizeof{\cycle} - 1$.
%         This result is a contradiction on Lemma~\ref{lem:m_onecycle}, as this cycle
%         can be solved individually in less than $\sizeof{\cycle} + 1$ swaps;
%         \item[] \textbf{Case (B):} If $\cycle_{\vertexu} \in \zeroCycles$ and 
%         $\cycle_{\vertexv} \in \oneCycles$, ;
%         \item[] \textbf{Case (C):} If $\cycle_{\vertexu} \in \zeroCycles$ and 
%         $\cycle_{\vertexv} \in \zeroCycles$, ;
%     \end{enumerate}
    
%     \end{itemize}
% \end{proof}

% The Lemma~\ref{lem:matching_inc_one} is going to help understand how the maximum
% matching behaves between each configuration when a swap is applied.
% In special, it will be used to show that the maximum number of swaps that can be saved
% from a cycle matching graph is exactly $2\times\sizeof{\matchingOf{\cycleMatchingGraph}}$.

\begin{lemma}
\label{lem:ub_cograph}
Let $\graphG$ be a cograph with an initial token placement $\initMap$.
Then, $\applyFunc{OPT_{\graphG}}{\initMap} \leq \sizeof{\vertexFunc{\graphG}} - 
\sizeof{\applyFunc{\oneCycles}{\conflictGraph_{\mapFunc}}} +
\sizeof{\applyFunc{\zeroCycles}{\conflictGraph_{\mapFunc}}} - 2\times\sizeof{\matchingOf
{\cycleMatchingGraph}}$.
\end{lemma}

\begin{proof}
Directly from Lemma~\ref{lem:m_onecycle}, Lemma~\ref{lem:m_zerocycle} and the previous discussion.
\end{proof}

\begin{lemma}
\label{lem:lb_cograph}
Let $\graphG$ be a cograph with an initial token placement $\initMap$.
Then, $\applyFunc{OPT_{\graphG}}{\initMap} \geq \sizeof{\vertexFunc{\graphG}} - 
\sizeof{\applyFunc{\oneCycles}{\conflictGraph_{\mapFunc}}} +
\sizeof{\applyFunc{\zeroCycles}{\conflictGraph_{\mapFunc}}} - 2\times\sizeof{\matchingOf
{\cycleMatchingGraph}}$.
\end{lemma}

\begin{proof}
Let $\applyFunc{p}{\graphG, \mapFunc} = \sizeof{\vertexFunc{\graphG}} - 
\sizeof{\applyFunc{\oneCycles}{\conflictGraph_{\mapFunc}}} +
\sizeof{\applyFunc{\zeroCycles}{\conflictGraph_{\mapFunc}}} - 2\times\sizeof{\matchingOf
{\cycleMatchingGraph}}$.
Note that $\applyFunc{p}{\graphG, \idMap} = 0$ holds.
First, it is going to be necessary to prove that for any swap on $\mapFunc$,
the value of $\applyFunc{p}{\graphG, \sub{\mapFunc}}$ will be decreased
by at maximum one, such that $\sub{\mapFunc}$ is a adjacent configuration
of $\mapFunc$.
The following cases encompass every possible split interaction between 
cycles.

\begin{itemize}
    \item \textbf{Case SPLIT-1:} Let $\vertexu$ and $\vertexv$ belong to 
    the same cycle $\cycle \in \oneCycles$. 
    This token swap breaks the original cycle into two vertex disjoint cycles that
    we will call $\cycle_{\vertexu}$ and $\cycle_{\vertexv}$.
    Assume that $\vertexu \in \vertexFunc{\cycle_{\vertexu}}$ and 
    $\vertexv \in \vertexFunc{\cycle_{\vertexv}}$.
    
    \begin{enumerate}
        \item[] \textbf{Case (A):} If $\cycle_{\vertexu} \in \oneCycles$ and 
        $\cycle_{\vertexv} \in \oneCycles$, the value of $\sizeof{\oneCycles}$ is
        increased in 1 and the size of the matching $\matchingOf{\cycleMatchingGraph}$
        does not change, as graph $\cycleMatchingGraph$ is not modified, resulting in
        $\applyFunc{p}{\graphG, \sub{\mapFunc}} = \applyFunc{p}{\graphG, \mapFunc} - 1$;
        \item[] \textbf{Case (B):} If  $\cycle_{\vertexu} \in \zeroCycles$ and 
        $\cycle_{\vertexv} \in \oneCycles$, the value of $\sizeof{\zeroCycles}$ is
        increased by 1, increasing the number of vertices in $\cycleMatchingGraph$ by
        one.
        By Lemma~\ref{lem:matching_inc_one}, the matching can increase in at most
        one, resulting in $\applyFunc{p}{\graphG, \mapFunc} - 1 \leq \applyFunc{p}
        {\graphG, \sub{\mapFunc}} \leq \applyFunc{p}{\graphG, \mapFunc} + 1$;
        \item[] \textbf{Case (C):} If $\cycle_{\vertexu} \in \zeroCycles$ and 
        $\cycle_{\vertexv} \in \zeroCycles$, the value of $\sizeof{\zeroCycles}$
        is increased by 2 and $\sizeof{\oneCycles}$ is decreased by 1, increasing
        the number of vertices in $\cycleMatchingGraph$ in two with a guaranteed additional
        edge between $\cycle_{\vertexu}$ and $\cycle_{\vertexv}$.
        By Lemma~\ref{lem:matching_inc_one}, this matching can increase in at most
        two, resulting in $\applyFunc{p}{\graphG, \mapFunc} - 1 \leq \applyFunc{p}
        {\graphG, \sub{\mapFunc}} \leq \applyFunc{p}{\graphG, \mapFunc} + 1$.
    \end{enumerate}
    
    \item \textbf{Case SPLIT-0:} Let $\vertexu$ and $\vertexv$ belong to 
    the same cycle $\cycle \in \zeroCycles$. 
    This token swap breaks the original cycle into two vertex disjoint cycles that
    we will call $\cycle_{\vertexu}$ and $\cycle_{\vertexv}$.
    Assume that $\vertexu \in \vertexFunc{\cycle_{\vertexu}}$ and 
    $\vertexv \in \vertexFunc{\cycle_{\vertexv}}$.
    
    \begin{enumerate}
        \item[] \textbf{Case (A):} Let $\cycle_{\vertexu},\cycle_{\vertexv} \in \oneCycles$.
        Then, the total number of swaps given by our model is $\sizeof{\cycle_{\vertexu}} - 1 + 
        \sizeof{\cycle_{\vertexv}} - 1 + 1 = \sizeof{\cycle_{\vertexu}} + 
        \sizeof{\cycle_{\vertexv}} - 1 = \sizeof{\cycle} - 1$.
        This result is a contradiction on Lemma~\ref{lem:m_onecycle}, as this cycle
        can be solved individually in less than $\sizeof{\cycle} + 1$ swaps;
        \item[] \textbf{Case (B):} If $\cycle_{\vertexu} \in \zeroCycles$ and 
        $\cycle_{\vertexv} \in \oneCycles$, the value of $\sizeof{\oneCycles}$
        is increased by one and $\sizeof{\zeroCycles}$ does not change.
        By Lemma~\ref{lem:zero_split}, $\lcaOf{\applyFunc{\conflictGraph}{\graphG}}
        {\vertexFunc{\cycle_{\vertexu} }} = \lcaOf{\applyFunc{\conflictGraph}{\graphG}}
        {\vertexFunc{\cycle}}$, not changing the graph $\cycleMatchingGraph$, resulting
        in $\applyFunc{p}{\graphG, \sub{\mapFunc}} = \applyFunc{p}{\graphG, \mapFunc} - 1$;
        \item[] \textbf{Case (C):} If $\cycle_{\vertexu} \in \zeroCycles$ and 
        $\cycle_{\vertexv} \in \zeroCycles$, the value of $\sizeof{\zeroCycles}$
        is increased by 1 and $\sizeof{\oneCycles}$ does not change.
        By Lemma~\ref{lem:zero_split}, either $\cycle_{\vertexu}$ or $\cycle_{\vertexv}$
        have the same lowest common ancestor as the cycle $\cycle$, not changing 
        the original related vertice on $\cycleMatchingGraph$ and adding exactly one
        additional vertice.
        Lemma~\ref{lem:matching_inc_one} show that the matching can increase in 
        at most one in this case, resulting in $\applyFunc{p}{\graphG, \mapFunc} - 1 
        \leq \applyFunc{p}{\graphG, \sub{\mapFunc}} \leq \applyFunc{p}{\graphG, 
        \mapFunc} + 1$.
    \end{enumerate}
    
\end{itemize}

It is also not needed to check the merge swaps in this case, as they are just the inverse
of a split.
This means that the equations have inverted plus and minus signals, and, as no equation
has other value greater than one, the proof that $\applyFunc{p}{\graphG, \mapFunc}$
decreases by at maximum one for every possible swap holds.
By the above analysis, it is possible to conclude that any token swap decreases
$\applyFunc{p}{\graphG, \mapFunc}$ by at most one for any token placement $\mapFunc$ and
obtain the Inequation~\ref{equ:decrease_by_one}.
    
\begin{equation}
    \centering
    \label{equ:decrease_by_one:cograph}
    \applyFunc{p}{\graphG, \sub{\mapFunc}} \geq \applyFunc{p}{\graphG, \mapFunc} - 1
\end{equation}

Thus, for any swapping sequence $\swapSeq = \createOrd{\swaps_1, \swaps_2, \ldots, 
\swaps_k}$ that transforms the initial configuration $\initMap$ to the identity
configuration $\idMap$ through adjacent configurations $\mapFunc_{1}, \mapFunc_2,
\ldots, \mapFunc_{k} = \idMap$, each pair of configurations $\applyFunc{p}{\graphG, 
\mapFunc_{j+1}} \geq \applyFunc{p}{\graphG, \mapFunc_{j}} -1$ holds from 
Inequation~\ref{equ:decrease_by_one:cograph} for $j = 1,2,\ldots,k-1$.
Take the sum of these inequations $\sum_{j} \applyFunc{p}{\graphG, \mapFunc_{j+1}} \geq \applyFunc{p}{\graphG, \mapFunc_{j}} -1$ shown in Inequation~\ref{equ:equation_sum:cograph}.
    
\begin{alignat}{2}
\centering
\label{equ:equation_sum:cograph}
&\applyFunc{p}{\graphG, \mapFunc_{1}}  \quad && \geq \quad \applyFunc{p}{\graphG, \mapFunc_{0}} -1 \nonumber \\
&\applyFunc{p}{\graphG, \mapFunc_{2}} \quad && \geq \quad \applyFunc{p}{\graphG, \mapFunc_{1}} -1 \nonumber \\
&\ldots \nonumber \\
&\applyFunc{p}{\graphG, \mapFunc_{k-1}} \quad && \geq \quad \applyFunc{p}{\graphG, \mapFunc_{k}} -1 \nonumber \\
 \cline{1-4}
&\applyFunc{p}{\graphG, \mapFunc_{k}} \quad && \geq \quad \applyFunc{p}{\graphG, \mapFunc_{0}} - \sizeof{\swapSeq}
\end{alignat}
    
By organizing Inequation~\ref{equ:equation_sum:cograph}, substituting $\applyFunc{p}{\graphG, 
\mapFunc_{k}}$ for a $0$, as it is the last configuration and this swap sequence solve the
instance, and substituting $\applyFunc{p}{\graphG, \mapFunc_{0}}$ by the original assumption,
the Inequation~\ref{equ:lower_bound:cograph} is found.
    
\begin{alignat}{2}
\centering
\label{equ:lower_bound:cograph}
& \applyFunc{p}{\graphG, \mapFunc_{k}} \quad && \geq \quad \applyFunc{p}{\graphG, \mapFunc_{0}} - \sizeof{\swapSeq} \nonumber \\
&\sizeof{\swapSeq} \quad && \geq \quad \applyFunc{p}{\graphG, \mapFunc_{0}} - \applyFunc{p}{\graphG, \mapFunc_{k}} \nonumber \\
&\sizeof{\swapSeq} \quad && \geq \quad \sizeof{\vertexFunc{\graphG}} - 
\sizeof{\applyFunc{\oneCycles}{\conflictGraph_{\mapFunc}}} +
\sizeof{\applyFunc{\zeroCycles}{\conflictGraph_{\mapFunc}}} - 2\times\sizeof{\matchingOf
{\cycleMatchingGraph}}
\end{alignat}

\end{proof}

\begin{theorem}
\label{cor:cograph_pol}
Let $\graphG$ be a cograph with an initial token placement $\initMap$.
The minimum number of required swaps is given by $\sizeof{\vertexFunc{\graphG}} - 
\sizeof{\applyFunc{\oneCycles}{\conflictGraph_{\mapFunc}}} +
\sizeof{\applyFunc{\zeroCycles}{\conflictGraph_{\mapFunc}}} - 2\times\sizeof{\matchingOf
{\cycleMatchingGraph}}$.
\end{theorem}

\begin{proof}
Directly from Lemma~\ref{lem:ub_cograph} and Lemma~\ref{lem:lb_cograph}.
\end{proof}

\begin{corollary}
Token Swap can be solved in polynomial time in cographs.
\end{corollary}

\begin{proof}
Directly from Theorem~\ref{cor:cograph_pol}.
\end{proof}

% Adicionar uma imagem aqui para representar a estrutura das classes que sobraram
% Complete Split
% Complete Bipartite


% \chapter{Calculating Upper-Bounds on the number of swaps in Cographs}

% \chapter{Algorithms for Minor Graph Classes}

\chapter{Integer Linear Programming Models}
\label{chp:ilp_models}

This chapter presents two integer linear programming models for the Token Swap
problem and the Parallel Token Swap problem, respectively, along with an 
explanation of each variable and constraint for each model.
% The chapter ends with an discussion on what is known about realizing token placement
% checks on a integer linear programming model.

\section{The TS Problem Formulation as an Integer Programming Problem}
\label{sec:formulation_ts}

The binary variables $\posVar$ determine if a token $\tokeni$ is at node $\vertexu$ 
in step $\stept$.
To correct model the bijections between tokens and vertices, Equation~\ref
{equ:model-ts:1} enforces that any token can be at most in one vertex and
Equation~\ref{equ:model-ts:2} that a vertex can have at most one token.
The binary variables $\swapVar$ flags if a swap happened between nodes
$\vertexu$ and $\vertexv$ in step $\stept$.
It is important to note that swaps are symmetric \textemdash i.e., a swap
$\createOrd{\vertexu, \vertexv}$ for $\vertexu, \vertexv \in \vertexSet$ is exactly 
the same as swap $\createOrd{\vertexv, \vertexu}$ \textemdash,
so variables $\swapVar$ and $\baseSwapVar{\vertexv}{\vertexu}{\stept}$ means 
the same swap.
On those grounds, and the fact that our graph is undirected, a technique that 
creates an ordering $\defineFunc{\vertexOrder}{\vertexSet}{\enum{\vertexNumber}}$ 
of the vertices of the graph and use the variables $\swapVar$ such that 
$\vertexu \lessVOrder \vertexv$ can be adopted, halving the number of variables 
needed to represent a swap.
The ordering $\vertexOrder$ is assumed to exist throughout the dissertation for any
graph.

\addExample{representation.tex}

The Figure~\ref{img:form1-example} is an example of a TS instance and a complete 
representation of an optimal swap sequence using $\swapVar$.
To retrieve the final sequence, it is necessary to traverse each matrix in step 
order and check each $\swapVar$, appending $\createOrd{\vertexu,\vertexv}$ to 
the swap sequence whenever $\swapVar = 1$.
Steps with no $\swapVar = 1$ can be ignored.
The total number of variables is given by Equation~\ref{equ:vars}.

\addEquation{ts-vars.tex}

The TS problem model given by Formulation~\groupRef{equ:model-ts:m}
{equ:model-ts:9} search viable solutions of the problem with a given 
upper-bound in the number of swaps $\swapBound$, allowing at most one swap 
per step $\stept \in \enum{\swapBound}$.
Each step is composed of a set of variables that describe the current 
configuration, which swap is being selected and Equation~\ref{equ:model-ts:7}
checks if a swap sequence solves the current instance.
The constant $\swapBound$ can be calculated by using any of the best 
approximation algorithms, or by using the trivial upper-bound $\worstCase{
\vertexNumber^{2}}$ on the size of an optimal swap sequence mentioned in Chapter
\ref{chp:intro}.

\addEquation{model-ts-cor.tex}

These approximation algorithms use the same intuitive lower-bound given by
$$\sum_{\vertexv \in \vertexSet} \frac{\graphDist{\initMap(\vertexv)}{\vertexv}}
{2}$$ where $\graphDist{\vertexu}{\vertexv}$ is the distance from vertex 
$\vertexu$ to vertex $\vertexv$ in the graph $\graphG$ and the division by $2$ 
happens since every swap can decrease the distance between tokens by at maximum 
two in the sum.
This bound is tight for instances of TS that can be solved with only swaps
that decrease the distance sum by two~\citep{Bonnet:2018}.
These values give us a good notion of the range of values that $\swapBound$ can
assume.
More detailed explanations of each constraint are presented in the subsequent  
Section~\ref{sec:formulation_ts:swaps}.

% \subsection{Modelling the Variables}
% \label{sec:formulation_ts:var}

% Binary variables $\posVar$ determine if a token $\tokeni$ is at node $\vertexu$ 
% in step $\stept$.
% To correct model the bijections between tokens and vertices, Equation~\ref
% {equ:model-ts:1} enforces that any token can be at most in one vertex and
% Equation~\ref{equ:model-ts:2} that a vertex can have at most one token.
% The binary variables $\swapVar$ flags if a swap happened between nodes
% $\vertexu$ and $\vertexv$ in step $\stept$.
% It is important to note that swaps are symmetric \textemdash i.e., a swap
% $\createOrd{\vertexu, \vertexv}$ for $\vertexu, \vertexv \in \vertexSet$ is exactly 
% the same as swap $\createOrd{\vertexv, \vertexu}$ \textemdash,
% so variables $\swapVar$ and $\baseSwapVar{\vertexv}{\vertexu}{\stept}$ means 
% the same swap.
% On those grounds, and the fact that our graph is undirected, a technique that 
% creates an ordering $\defineFunc{\vertexOrder}{\vertexSet}{\enum{\vertexNumber}}$ 
% of the vertices of the graph and use the variables $\swapVar$ such that 
% $\vertexu \lessVOrder \vertexv$ can be adopted, halving the number of variables 
% needed to represent a swap.
% The ordering $\vertexOrder$ is assumed to exist throughout the dissertation for any
% graph.

% \addExample{representation.tex}

% The Figure~\ref{img:form1-example} is an example of a TS instance and a complete 
% representation of an optimal swap sequence using $\swapVar$.
% To retrieve the final sequence, it is necessary to traverse each matrix in step 
% order and check each $\swapVar$, appending $\createOrd{\vertexu,\vertexv}$ to 
% the swap sequence whenever $\swapVar = 1$.
% Steps with no $\swapVar = 1$ can be ignored.
% The total number of variables is given by Equation~\ref{equ:vars}.

% \addEquation{ts-vars.tex}

\subsection{Modelling Swaps and Initial Token Configuration}
\label{sec:formulation_ts:swaps}

The constraints~\groupRef{equ:model-ts:3}{equ:model-ts:6} are being used to 
model the behavior of swaps.
For a swap $\swapVar$ to be possible, it is required that in step $\stept$ 
the token on vertex $\vertexu$ is on vertex $\vertexv$ on step $\stept+1$, 
given by Inequality~\ref{equ:model-ts:3}.
Similarly, on Inequality~\ref{equ:model-ts:4}, it was guaranteed that the second 
token that is on vertex $\vertexv$ on step $\stept$ must be on vertex 
$\vertexu$ on step $\stept+1$.
Note that each pair of vertices $\vertexu,\vertexv \in \vertexSet$ were 
directly chosen from the edge set $\edgeSet$, ensuring that an edge must exist.
Inequality~\ref{equ:model-ts:5} is there to keep the consistency between each 
pair of vertices that do not have an edge in between, as it is impossible to have 
a token traverse both in one step.
Inequality~\ref{equ:model-ts:sym} restrict the symmetry of the model, eliminating
steps with no swaps followed by a step with a swap.
Inequality~\ref{equ:model-ts:6} force each step to have at maximum one swap, 
while also permitting steps with no swaps.

The Inequality~\ref{equ:model-ts:ini} creates the necessary constraints
to represent the initial configuration, setting the necessary variables
to one.
The step $\stept = 0$ is fixed on the model for any given configuration
$\initMap$, as there must exist one token on each vertice initially.
In Section~\ref{sec:formulation_pts} a more in-depth discussion about a 
parallel variant of the TS problem that do not need to guarantee one 
swap per step will be presented.

\section{The Parallel TS Problem Formulation as an Integer Programming Problem}
\label{sec:formulation_pts}

The {\texttt Parallel Token Swap} (PTS) problem is a version of TS where swaps can 
be applied in parallel.
For a graph $\graphG \defeq (\vertexSet, \edgeSet)$, a \textit{parallel swap} of 
a set of parallel swaps $\swapPSeq = \createSet{\swaps_1, \swaps_2, \ldots, 
\swaps_k}$ is a swap $\swaps$ such that no other swap in $\swapPSeq$ shares a 
vertex with $s$.
Any swap sequence $\swapSeq$ that solves an instance of PTS or TS can be 
partitioned into sequences of parallel swaps $\applyFunc{\pazocal{\swapPSeq}}{S} 
= (\swapPSeq^{0}, \swapPSeq^{1}, \ldots, \swapPSeq^{b})$.
The swaps of each partition can be applied to a token placement in any order 
without changing any intermediate step in $\swapSeq$ (hence the use of set) or 
it can be applied by just one representation function $\mapFunc_{\swapPSeq}$, as 
all swaps in a partition have disjoint vertices.
Equation~\ref{equ:parswaps} shows how this representation function can be built.
A partitioned sequence of parallel swaps solves an instance of TS or TSP if and 
only if the identity function is resulted by applying each partition 
iteratively, while the swap application order of each partition does not matter.

\addEquation{partition.tex}

It is important to note that the size of any partition has a natural upper-bound 
on the size of the {\texttt Maximum Matching} of the graph $\graphG$, as it is the
maximum number of edges that can be swapped in the graph and still be disjoint.
Calculating the maximum matching size in general graphs can be done in 
polynomial time~\citep{Edmonds:1965}.
Therefore, for graphs with maximum matching size of 1, an instance of PTS has the
same set of optimal swap sequences of the equivalent TS instance.
For the general case, the size of a swap sequence that solves a TS instance can be
used as an upper-bound for an equivalent PTS instance.
This variant of the TS problem has a deeper relation to parallel sorting on SIMD
machines consisting of several processors with local memory connected by a 
network~\citep{Yamanaka:2015,Kawahara:2017} and is better suited for some 
reconfiguration problems, like Qubit Allocation~\citep
{Siraichi:2018,Siraichi:2019}.

One interesting question that naturally arises from this definition is: "what 
is the swap sequence that solves a given PTS instance and minimizes the number 
of partitions in $\sizeof{\applyFunc{\pazocal{\swapPSeq}}{\swapSeq}}$?".
In the decision version, the existence of at least one swap sequence that solves 
a PTS instance with $k$ or less partitions is wanted.
This version of the problem has already been proven to be \textbf{NP-Complete}
for general $k$, but can be calculated in polynomial time if $k \leq 2$ and
have an approximation algorithm for paths~\citep{Kawahara:2017}.

Each step is now related to a partition and a new binary variable for each step 
called $\stepVar$ were added to keep track of which steps have \textit{at least} 
one $\swapVar = 1$ by using Inequality~\ref{equ:model-pts:7}.
Note that the number of checks is improved by only verifying variables $\swapVar$
such that $\createSet{\vertexu,\vertexv} \in \edgeSet$, as swaps can only exist on the 
edges.
These flags help counting how many steps are using at least one swap, while any 
step that do not need to be used to achieve a minimal swap sequence will not be 
counted, as $\stepVar = 0$.
This variable and the fact that each step is now allowed to have any number of
parallel swaps allows for the minimization of $\stepVar$ in the minimization 
rule given by Equation~\ref{equ:model-pts:m}, which is equivalent to minimizing 
the number of partitions.
The total number of variables is increased by $\swapBound$ from Equation
\ref{equ:vars}.

\addEquation{model-pts.tex}

The model presented by Formulation~\groupRef{equ:model-pts:m}{equ:model-pts:11} 
is an adaptation of the TS model for the PTS problem. 
The variables $\swapVar$ and $\posVar$ are used in exactly the same way as the 
TS model with constraints~\groupRef{equ:model-pts:1}{equ:model-pts:5} being 
exactly the same as constraints~\groupRef{equ:model-ts:1}{equ:model-ts:5}.
Inequalities~\ref{equ:model-pts:sym} and \ref{equ:model-pts:ini} are respectively 
equivalent to the Inequalities~\ref{equ:model-ts:sym} and \ref{equ:model-ts:ini}.
Each new constraint will be explained in detail in Section \ref{sec:formulation_pts:par}.

% \subsection{Minimizing the Number of Steps}
% \label{sec:formulation_pts:min}

% Each step is now related to a partition and a new binary variable for each step 
% called $\stepVar$ were added to keep track of which steps have \textit{at least} 
% one $\swapVar = 1$ by using Inequality~\ref{equ:model-pts:7}.
% Note that the number of checks is improved by only verifying variables $\swapVar$
% such that $\createSet{\vertexu,\vertexv} \in \edgeSet$, as swaps can only exist on the 
% edges.
% These flags help counting how many steps are using at least one swap, while any 
% step that do not need to be used to achieve a minimal swap sequence will not be 
% counted, as $\stepVar = 0$.
% This variable and the fact that each step is now allowed to have any number of
% parallel swaps allows for the minimization of $\stepVar$ in the minimization 
% rule given by Equation~\ref{equ:model-pts:m}, which is equivalent to minimizing 
% the number of partitions.
% The total number of variables is increased by $\swapBound$ from Equation
% \ref{equ:vars}.

\subsection{Modelling Parallel Swaps}
\label{sec:formulation_pts:par}

To allow parallel swaps in the model, an efficient way to check if two swaps
are disjoint must be used.
If two swaps are not disjoint, then they have exactly one vertex in common, as 
it is impossible to have repeated swaps in the same step due to how the variables 
$\swapVar$ are modeled.
Thus, for any two conflicting swaps, a combination of three nodes $\vertexu, 
\vertexv, \vertexw$ could be taken and assumed, without loss of generality, that
$\vertexw$ is part of these two swaps $\createOrd{\vertexu, \vertexw}$, 
$\createOrd{\vertexw, \vertexv}$.
Consequently, the variables $\baseSwapVar{\vertexu}{\vertexw}{\stept}$ and 
$\baseSwapVar{\vertexw}{\vertexv}{\stept}$ can not be $1$ at the same time
in the model, as given by Inequality~\ref{equ:model-pts:6}.
In the case that both edges exist in the graph of an instance, these swaps would 
share a vertex and could not be used in the same swap.
On the cases that these vertices have only one edge or no edge, there is no
way two swaps can be used between these three vertices, as Inequalities~\groupRef
{equ:model-pts:3}{equ:model-pts:5} prohibit them.

\addExample{pairings.tex}

The concept was improved by the use of the inherent vertex ordering 
$\vertexOrder$ of the graph to remove repeated vertex checks from the model.
For any vertex triple $\vertexu$, $\vertexv$, $\vertexw$ with a "center"
vertex fixed as $\vertexw$, checking variables $\baseSwapVar{\vertexu}
{\vertexw}{\stept}$ and $\baseSwapVar{\vertexw}{\vertexv}{\stept}$ are 
equivalent to checking variables $\baseSwapVar{\vertexv}{\vertexw}{\stept}$ and 
$\baseSwapVar{\vertexw}{\vertexv}{\stept}$.
Therefore, checking variables for triples $\vertexu$, $\vertexw$, $\vertexv$ 
such that $\uvComp$ is enough and remove useless checkings.
Figure~\ref{img:form2-example} give us an example to compare swap sequence 
representations between TS and PTS models.

% \section{Observations on Realizing Token Placements Checks}
% \label{sec:obs}

% Let $\mapFunc$ be a token placement map of a TS instance.
% The sum of the distances $\graphDist{\applyFunc{\mapFunc}{\vertexu}}{\vertexu}$
% is the sum of the distances between each token to its target vertex. 
% With the sum, one could test if a swap sequence $\swapSeq$ solves the instance
% by checking if is 0 for the resulting placement $\mapFunc_{\sizeof{\swapSeq}}$.
% For trees, every swap can be classified in one of three categories related to 
% the sum of distances of an instance: (a) The swap decreases the sum by two 
% through moving two tokens closer to its target vertices, also called a 
% \textit{happy swap}; (b) the swap does not change the total sum by moving one 
% token closer and one token further from their target vertices and (c) the swap 
% increases the sum by two, as it moves two tokens further from its target 
% vertices.

% Intuitively, one could think that any swap sequence that solves a TS instance 
% with swaps restricted to categories (a) and (b) will have less swaps than a 
% swap sequence that solves the same instance and uses swaps of category (c) 
% \textemdash as \citep{Smith:1999} tried to prove, but subsequently found an 
% error~\citep{Smith:2011:Corrigendum}.
% Then, \citep{Vaughan:1991} conjectured that any optimal swap sequence would 
% not swap already correct tokens on leafs and called them \textit{happy leafs}, 
% resulting in the so-called \textit{Happy Leaf Conjecture}\footnote{The original 
% paper actually claimed a stronger conjecture with the Happy Leaf Conjecture as 
% a special case.}.
% This conjecture was disproved by \citep{Ahmad:2019}, as they shown that there is 
% a class of infinite trees that necessarily need (c) swaps to achieve the optimal 
% number of swaps and that any algorithm that do not consider swaps on 
% happy leafs has an approximation factor of \textit{at least} $\frac{4}{3}$ for 
% general graphs and trees.
% The two approximative algorithms cited in Section~\ref{chp:intro} can only 
% generate swap sequences of category (a) and (b).

% \addEquation{dist-sum.tex}

% For a given step $\stept$, it was presumed that there is no guarantee that an 
% adjacent placement will increase or decrease the sum in step $\stept+1$.
% In such manner, constraints that keeps track of this sum in every intermediate 
% placement do not look viable for general graphs and trees.
% Equation~\ref{equ:model-ts:8} and Equation~\ref{equ:model-pts:8} were chosen due
% to the simplicity of pushing the placement verification to the furthest step 
% possible, when $\stept = \swapBound$, and allowing the existence of steps of any 
% swap category.
% Let $d_{\vertexu\vertexv} \in \mathbb{N}_0$ be a constant that represents the 
% distance between any two vertices $\vertexu,\vertexv \in \vertexSet$. 
% If there is a class of instances that can be proved to not have swaps of 
% category (c), the Inequality~\ref{equ:plc-sum} could be added to the model to 
% cut off token placements that increase the sum of the distances, but the 
% tradeoff of pre-calculating this distance would have to be considered.

\chapter{Conclusion}

This dissertation presented the Token Swap problem and one parallel variant, 
the Parallel Token Swap problem.
Initially, the work is focused on introducing the necessary mathematical tools
for constructing the subsequent proofs, with special emphasis on the crucial 
notion of \textit{merge} and \textit{split} swaps.
Then, these tools are used to build a method for finding an optimal
swap sequence for the class of threshold graphs.
From this method, a new method is derived to also find an optimal swap sequence
for the class of cographs.

Then, to go along with these results, an initial integer linear model for each
of the problems is presented together with a detailed discussion about the 
constraints.
The next step would be the coding and solving each of these models to 
check their viability and compare with other approaches.
Moreover, these models could be adapted to other TS variations, such as the 
Colored Token Swap problem and the Parallel Colored Token Swap
Generalizations of the TS problem and PTS problem, respectively, where tokens
can have more than one target vertex.

% The presented models could still be improved with testing.
% More constraints would focus on the development of more valid inequalities, like 
% checking steps with repeated configurations or constraints to eliminate the 
% token symmetry of the model, pushing steps with no swaps to the rightmost step
% possible. Some of those were already achieved, but omitted for the sake of 
% conciseness. 
% On future works, with experimental results, we will evaluate whether they 
% provide improvements in running time.
% Another route would be the creation of specialty models focused in some useful 
% and hard classes of graphs like trees, as more specific constraints could be 
% helpful in getting faster answers.

On future works the mathematical notions presented in this dissertation could 
be used to understand and build different methods for other graph classes.
Moreover, the presented integer linear models could still be improved with 
implementation and testing, as they are an initial model.
More constraints would focus on the development of more valid inequalities to
optimize the model.
Another route would be the creation of specialty models focused in some useful 
and hard classes of graphs like trees, as more specific constraints could be 
helpful in getting faster answers.

\renewcommand\bibname{References} %% Trabalhos em inglês
\bibliographystyle{plain}
\bibliography{bibfile}


% Este comando encapsula o conjunto de apêndices. A sua função é fazer com que
% a numeração dos apêndices seja feita com letras maiúsculas (A, B, C, etc.) e
% a palavra "Apêndice" anteceda as entradas no Sumário.
\begin{appendices}
\end{appendices}


\end{document}
