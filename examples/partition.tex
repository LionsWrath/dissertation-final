\begin{figure}[H]
    \centering
    \begin{tikzpicture}[node distance=2cm, 
        decoration={
            markings,
            mark=at position 0.5 with {\arrow{>}}}
    ]

        % f
        \node[] (as1) at (1, 3) [] {$a$};
        \node[] (bs1) at (1, 2.5) [] {$b$};
        \node[] (cs1) at (1, 2) [] {$c$};
        \node[] (ds1) at (1, 1.5) [] {$d$};
        \node[] (es1) at (1, 1) [] {$e$};
        \node[] (fs1) at (1, 0.5) [] {$f$};

        \node[] (as2) at (2, 3) [] {$a$};
        \node[] (bs2) at (2, 2.5) [] {$b$};
        \node[] (cs2) at (2, 2) [] {$c$};
        \node[] (ds2) at (2, 1.5) [] {$d$};
        \node[] (es2) at (2, 1) [] {$e$};
        \node[] (fs2) at (2, 0.5) [] {$f$};

        % P1
        \node[] (ar1) at (3.5, 3) [] {$a$};
        \node[] (br1) at (3.5, 2.5) [] {$b$};
        \node[] (cr1) at (3.5, 2) [] {$c$};
        \node[] (dr1) at (3.5, 1.5) [] {$d$};
        \node[] (er1) at (3.5, 1) [] {$e$};
        \node[] (fr1) at (3.5, 0.5) [] {$f$};

        \node[] (ar2) at (4.5, 3) [] {$a$};
        \node[] (br2) at (4.5, 2.5) [] {$b$};
        \node[] (cr2) at (4.5, 2) [] {$c$};
        \node[] (dr2) at (4.5, 1.5) [] {$d$};
        \node[] (er2) at (4.5, 1) [] {$e$};
        \node[] (fr2) at (4.5, 0.5) [] {$f$};

        % P2
        \node[] (af1) at (  6, 3) [] {$a$};
        \node[] (bf1) at (  6, 2.5) [] {$b$};
        \node[] (cf1) at (  6, 2) [] {$c$};
        \node[] (df1) at (  6, 1.5) [] {$d$};
        \node[] (ef1) at (  6, 1) [] {$e$};
        \node[] (ff1) at (  6, 0.5) [] {$f$};

        \node[] (af2) at (  7, 3) [] {$a$};
        \node[] (bf2) at (  7, 2.5) [] {$b$};
        \node[] (cf2) at (  7, 2) [] {$c$};
        \node[] (df2) at (  7, 1.5) [] {$d$};
        \node[] (ef2) at (  7, 1) [] {$e$};
        \node[] (ff2) at (  7, 0.5) [] {$f$};
       
        % Edges
        \draw

            % Map f
            (as1.east) edge[c5, ->, line width=0.25mm] (bs2.west)
            (bs1.east) edge[c2, ->, line width=0.25mm] (ds2.west)
            (cs1.east) edge[c1, ->, line width=0.25mm] (es2.west)
            (ds1.east) edge[c3, ->, line width=0.25mm] (as2.west)
            (es1.east) edge[c4, ->, line width=0.25mm] (cs2.west)
            (fs1.east) edge[c6, ->, line width=0.25mm] (fs2.west)

            % P1
            (ar1.east) edge[c5, ->, line width=0.25mm] (br2.west)
            (br1.east) edge[c2, ->, line width=0.25mm] (dr2.west)
            (cr1.east) edge[c1, ->, line width=0.25mm] (cr2.west)
            (dr1.east) edge[c3, ->, line width=0.25mm] (ar2.west)
            (er1.east) edge[c4, ->, line width=0.25mm] (er2.west)
            (fr1.east) edge[c6, ->, line width=0.25mm] (fr2.west)

            % P2
            (af1.east) edge[c3, ->, line width=0.25mm] (af2.west)
            (bf1.east) edge[c5, ->, line width=0.25mm] (bf2.west)
            (cf1.east) edge[c1, ->, line width=0.25mm] (ef2.west)
            (df1.east) edge[c2, ->, line width=0.25mm] (df2.west)
            (ef1.east) edge[c4, ->, line width=0.25mm] (cf2.west)
            (ff1.east) edge[c6, ->, line width=0.25mm] (ff2.west);

        % 
        \draw ( 0.8,0.2) rectangle ( 2.2, 3.2);
        \draw ( 3.3,0.2) rectangle ( 4.7, 3.2);
        \draw ( 5.8,0.2) rectangle ( 7.2, 3.2);

        % Write data
        \node at (1.5,3.5) {$\mapFunc$};
        \node at (  4,3.5) {$\mapFunc_{\partitionElem_{1}}$};
        \node at (6.5,3.5) {$\mapFunc_{\partitionElem_{2}}$};

        % Write equal
        \node at (2.75,1.75) {$=$};
        \node at (5.25,1.75) {$\circ$};

    \end{tikzpicture}
    \caption{Let $\mapFunc$ be a valid token configuration and $\partitionFunc{\createSet{a,b,c,d,e,f}} = \createSet{\createSet{a,b,d}, \createSet{c,e}, \createSet{f}}$ be a partition of $\mapFunc$. This partition is valid, as elements $\mapFunc_{\partitionElem_{1}}$, $\mapFunc_{\partitionElem_{2}}$ and $\mapFunc_{\partitionElem_{3}}$ are a valid token configurations. Moreover, this partition is also the coarsest partition, as there is no other partition of greater size such that every element are valid. The above example shows that the composition operation of the elements of the partition returns to the original mapping. Note that the mapping $\mapFunc_{\partitionElem_{3}}$ is equivalent to the identity configuration and not shown.}
    \label{img:function_partition}
\end{figure}
