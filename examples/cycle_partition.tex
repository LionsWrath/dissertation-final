\begin{figure}[H]
    \centering
    \begin{tikzpicture}[node distance=2cm, 
        outer/.style={circle,draw,inner sep=1pt, minimum size=3.5mm}, 
        inner/.style={rectangle,draw},
        lca/.style={inner,fill=gray-line}
    ]
        % Inner Nodes
        \node[lca] (a) at ( 0, 4) [] {$1$};
        \node[inner] (b) at ( 1.5, 3) [] {$0$}; % aqui
        \node[lca] (c) at ( -1, 3) [] {$0$};
        \node[inner] (f) at ( 0.5, 2) [] {$1$}; % aqui
        \node[inner] (g) at ( -1, 2) [] {$1$};
        
        % Outer Nodes
        \node[outer,c2,font=\tiny] (h) at ( -2, 2) [] {$1$};
        
        \node[outer,c2,font=\tiny] (i) at ( -1.25, 1) [] {$2$};
        \node[outer,c2,font=\tiny] (j) at ( -0.75, 1) [] {$3$};
        
        \node[outer,c4,font=\tiny] (k) at ( 0.25, 1) [] {$4$}; % aqui
        \node[outer,c4,font=\tiny] (o) at ( 0.75, 1) [] {$5$}; % aqui
        
        \node[outer,c4,font=\tiny] (n) at ( 1.25, 2) [] {$6$}; % aqui
        \node[outer,c4,font=\tiny] (m) at ( 1.75, 2) [] {$7$}; % aqui
        
        % Edges
        \draw
            % Inner Edges
            (a.south) edge (b.north)
            (a.south) edge (c.north)
            (b) edge (n)
            (b) edge (m)
            (c.south) edge (f.north)
            (c.south) edge (g.north)
            (c.south) edge (h.north)
            (g) edge (i)
            (g) edge (j)
            (f) edge (k)
            (f) edge (o);
        
        \draw [draw,dashed] (-0.5,0.7) rectangle (-2.3,3.4);
        \draw [draw,dashed] (-0.4,0.7) rectangle (2,4.4);

        \useasboundingbox (0,0) rectangle (2,2.5);
        
        \node at ( 1.7,0.4) [] {\textsc{$\cycle_{2}$}};
        \node at (-2,3.7) [] {\textsc{$\cycle_{1}$}};
            
    \end{tikzpicture}
    
    \caption{The cotree of the graph showed in Figure~\ref{img:cograph} is being used. 
    Let $\mapFunc$ be a configuration where nodes 1,2,3 are part of a permutation cycle 
    $\cycle_{1}$ and nodes 4,5,6,7 are part of another permutation cycle $\cycle_{2}$, 
    without paying attention to the exact cycle configuration. 
    The lowest common ancestor of each cycle is denoted as a gray rectangle inside each 
    corresponding labeled rectangle.
    The respective partitions are $\partitionFunc{\cycle_{1}} = \createSet{\createSet{2,3},
    \createSet{1}}$ and $\partitionFunc{\cycle_{2}} = \createSet{\createSet{4,5},
    \createSet{6,7}}$ respectively.
    }
    \label{img:cycle_partition}
\end{figure}