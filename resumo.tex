O framework de reconfiguração introduz o conceito de
transformação em problemas computacionais, mostrando 
novas preocupações como resultado da necessidade de
compreender estas mudanças sob uma variedade de operações 
e restrições.

Quando se trata de desafios de reconfiguração, os
três parâmetros de importância são conexão, diâmetro, 
e distância. 
Esta dissertação se concentra no Token Swap, um problema 
de reconfiguração onde o objetivo é converter uma 
configuração inicial de fichas de um grafo em uma 
configuração identidade de fichas que mapeia cada ficha 
para seu vértice com a menor distância possível.

O principal resultado dessa dissertação é a construção das 
ferramentas matemáticas necessárias e da prova de existência 
de um algoritmo ótimo para grafos da classe \textit{threshold} 
e, subsequentemente, cografos. 
Então, alguns trabalhos preliminares sobre modelos de 
programação linear inteira para os problemas de \textit{Token Swap} 
e \textit{Parallel Token Swap} também são apresentados, juntamente 
com o raciocínio por trás de cada restrição.

\keywords{Teoria dos Grafos, Problemas de Reconfiguração, Pesquisa Operacional}
