The reconfiguration framework introduces the concept of 
transformation into computational issues, posing new 
concerns as a result of the need to comprehend these 
changes under a variety of operations and constraints.

When it comes to reconfiguration challenges, the three 
parameters of importance are connection, diameter,
and distance.
This dissertation focuses on the Token Swap problem, 
a reconfiguration problem where the goal is to convert 
an initial token placement on a graph into an identity 
token placement that maps every node to itself with the 
shortest distance possible.

The main result of this dissertation is the construction
of the necessary mathematical tools and the proof of
existence of a optimal algorithm for the class of threshold
graphs and subsequently cographs.
Then, some preliminary work on integer linear programming
models for the problems of Token Swap and Parallel Token Swap 
will also be presented, together with a simple reasoning
behind each constraint.